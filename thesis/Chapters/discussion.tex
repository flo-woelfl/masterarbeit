

\chapter{Conclusion}


\section{Encountered Problems}

Even though SPECFEM3D is relatively well documented compared to other solvers I used in 
earlier works, there are still some parts of the software that lack sufficient documentation.
For example multiple simultaneous runs of several earthquake events are possible with the 
software, however they are not mentioned in the manual.
Therefore the required files and directory structure from the error messages of
several runs had to be figured out. 
Furthermore more advanced functionalities like the creation of misfit kernels from event kernels are
possible with the SPECFEM utility programs, but sparsely or not at all documented in the manual.
Another challenge was to find the correct time-shift needed for the adjoint simulations, as this 
one is different from the forward simulations, where no time-shift is applied.

It was planned to run all simulations with the option to undo attenuation effects exactly. 
This was no problem for the forward simulations, however it caused the adjoint kernel
simulations to run out of memory.
This problem is mentioned in the SPECFEM Manual and one possible reason is not exclusive use
of compute nodes. 
In this work the UNDO\_ATTENUATION parameter is used for the forward simulations to compute the 
misfits and the resulting adjoint sources as exactly as possible.
The adjoint kernel are then run with only partial physical dispersion enabled.

\section{Review of LASIF with SPECFEM3D GLOBE} 

One of the objectives of this work is to test how well LASIF interacts with the SPECFEM software,
as it has mostly been tested with SES3D before.
LASIF proves to be a large time-saver for the preparation of the simulations. 
The bulk of my observed data was downloaded with the download tools included in LASIF, which 
has the big advantage that all raypaths are checked if they are within the domain and can 
directly be deleted otherwise.
Only a few of the GSN seismic data had to be acquired from the IRIS webservices.
That the domain of this work covers the South Pole was an extra challenge, as the LASIF domain
are defined at the equator and then rotated to the center of the actual domain.
Lion Krischer did change the domain definitions, so that domains covering the poles  are now 
possible as well. 
The SPECFEM mesh is defined a little bit different at the margins of the domain, so that it 
covered additional area. However these effects are small enough to be neglected.
The processing of observed and synthetic data within LASIF worked well after a few changes to
the processing scripts.
As a large part of my synthetic seismograms are quite similar to their observed counterparts,
the window selection could be done mostly automatically. 
To select additional wiggles, especially the P-wave arrivals the LASIF misfit GUI proved helpful.
After small changes required for SPECFEM's naming convention the resulting adjoint source files
can be used without a problem in the solver.
So it can be concluded that LASIF can be used for SPECFEM adjoint simulations and helps to save a 
lot of time compared to manual organisation of the whole project.


\section{Next Steps}

With the summed up sensitivity kernel from all 50 events the initial velocity
model can be updated as described in the section~\ref{Theory_Adjoint}.
This has to be done iteratively until the model update $\delta \boldsymbol{m}$
converges to a predefined small value.
When no more model updates can be obtained the resulting new velocity model 
can then be integrated in the global CSEM velocity model.
The final model can then be interpreted for geological structures like 
under the Transantarctic Mountains and spreading zones surrounding Antarctica.

\vspace*{1cm}
