%*******************************************************
% Introduction
%*******************************************************
% \pdfbookmark[0]{Introduction}{introduction}

\lhead{\emph{Introduction}}

\chapter{Introduction}

\section{Objective of this Thesis} % thesis in capital letters?

The objective of this work is to perform the initial steps needed for the improvement of a continental scale tomographic model for the crustal and upper mantle structure.
% For this a starting model like S20RTS within the Comprehensive Earth Model is used.
% by means of full seismic waveform inversion using the adjoint technique. 
Long period seismic signals of 60 to 120 seconds % check the numbers again
are used for the first iterations of the forward model. 
The limitation to only low frequencies and %number of iterations
iterations is due to the restricted time for the Master's thesis.
% The model updates are done by simulating a forward model and with its result an adjoint model. % improve this sentence
Testing the Large Scale Seismic Inversion Framework (LASIF) in general and with respect to its compatibility with SPECFEM3D 
GLOBE is a further intention of this work.
% The preprocessing and preparation for the adjoint simulations is performed with LASIF. 
The measured waveforms are dominated by surface waves. 
% are used for the tomography as 
They are more suitable for large upper mantle structures than body waves, as they are normally sensitive to a depth of a few 
hundred kilometers \citep{Morelli2004}.

\section{Overview of the Thesis}
In the following thesis the first steps of a seismic tomography with the full waveform inversion
(FWI) 
are described.
Reasons why Antarctica is chosen as region of interest are specified and the unique geologic 
situation of the southernmost continent is explained.
The framework LASIF and its tools to download and process the required 
waveforms are introduced.
Furthermore the Spectral Element Method (SEM) that builds the basis of the SPECFEM3D GLOBE 
solver is depicted.
The theoretical background of the FWI is shortly discussed and how model improvements 
of an initial Earth velocity model can be achieved with the adjoint method.
The Comprehensive Seismic Earth Model used for the simulations is shown with
respect to differences to one-dimensional models for the Antarctic region.
The workflow of the simulations consisting of the mesh creation, forward simulations, 
misfit determination, computation of adjoint sources and the adjoint kernel simulations
are described.\\
Subsequently the resulting event and misfit kernels are presented and explained.
Encountered problems with the solver are discussed and possible next steps are proposed.
The thesis ends with a tutorial on how to perform adjoint kernel simulations
with LASIF and SPECEM3D GLOBE for future users.


\section{Short Overview of Seismic Tomography}

Tomography in seismology started with simple ray based methods.
The traveltime of one phase (often the first arrival) from an observed earthquake
is compared with the simulation with a one-dimensional model.
The first well-known one of the 1D models was the Preliminary Reference Earth Model
(PREM) by \citealp{dziewonski1981preliminary}.\\
%
Later with higher computing power multiple phases and surface waves were included
to improve the quality of the inversions with finite frequency approaches.
Different 3D tomographies agree relatively well for long wavelengths, but much less
for higher frequencies \citep{Liu2012}. \\
%
Full waveform inversion are used for a long time in exploration geophysics;
this method does not require the picking of individual seismic phases.
As a larger part of the seismogram is used it is especially well suited
for regions with sparse data.
This is one of the reasons why FWI is chosen for this work.

\section{Suitability of Antarctica}

There are several reasons why the Antarctic continent is a suitable and interesting region 
to study with seismic tomography:
\begin{enumerate}[i]
\item All other continents are already better studied with seismic tomographies.
\item In recent years more stations were installed on the continent so that the availability 
of seismic waveforms increased.
\item The coverage of Antarctica is quite well as the continent is surrounded by oceanic spreading zones 
and stations are installed on the surrounding continents.
\item The geology of Antarctica is not well known and a seimic tomography could give more information
for example on the orogenesis of the Transantarctic Mountains.
\end{enumerate}


\section{Geology of Antarctica}
Antarctica is an appropriate choice for a seismic tomography study as it is the least seismically explored continent 
due to its remoteness and being surrounded by large oceans. 
Its lithosphere features some one-of-a-kind geological properties. 

Tectonically Antarctica is divided in two parts by the Transantarctic Mountains (TAM) spanning from the Ross Sea to the Weddel Sea.
The two sections are the West Antarctic Rift System (WARS) and the much larger and geologically older East Antarctic (EA) craton that 
was part of the Gondwana supercontinent \citep{Gupta2009}. 
% more on WARS (one of the largest continental rift zones) , or TAM not caused by contraction or subduction, ... from Morelli and references therein
The highest mountains in the TAM are up to 4500$\,$m high and the mountain chain is circa 3500$\,$km long \citep{Morelli2004}.
Locations of the two parts and the TAM can be seen in \autoref{ant-map}.
% origin of TAM 
The orogenesis % is this term here correct?
of the TAM is disputed, possible reasons are combinations of thermal uplift due to its proximity of the WARS continental rift, 
isostatic rebound after unloading of ice masses, flexural uplift after the breakup of two tectonic plates or 
the remains of a larger high plateau \citep{VanWijk2008}.
Model results from \citealp{VanWijk2008} suggest that the most likely reason for the uplift of the TAM is an extension along the 
boundary between the EA and Western Australia. This could also explain the origination of the nearby Wilkes depression.
Furthermore hotspots have be suggested to exist in Antarctica like under the active Erebus volcano \citep{Gupta2009}.

The size of the continent (larger than Europe) makes long period tomography suitable.  
Seismological tomography of Antarctica started with the construction of global models of wave velocity like \citealp{Woodhouse1984}.
It is known from previous tomographic analysis that the older, eastern part of the continent consists of seismically faster material than 
the younger Rift System in the West \citep{Morelli2004}.

Seismic research on Antarctica was for a long time limited by the few available seismometers at research stations on the continent. 
The worldwide installation of the GEOSCOPE network let to first maps of Antarctica's lateral heterogeneities \citep{Roult1994}.
Before 2007 seismic studies relied on global networks or temporarily installed seismometer networks, the most notable
example is the TAMSEIS experiment \citep{Lawrence2006}. 
After 2007 permanent broadband stations were installed within the Gamburtsev Antarctic Mountains Seismic Experiment (GAM-SEIS/AGAP)
as well as the Polar Earth Observing Network (POLENET ANET); 
in total there are now 77 year-round seismic stations available \citep{Anthony2014}.  




% maybe more on concepts of inversion and tomography
 
% Why FWI was chosen
% simple classical ray method -> finite frequency methods -> FWI (fully numerical)


\begin{figure}[h]
\begin{center}
\includegraphics[width = 1\textwidth]{images/antarctica-map_edit.png}
\caption[Map of Antarctica \url{http://lima.nasa.gov/pdf/A3_overview.pdf}.]
{This map shows the West Antarctic Rift System the Transantarctic Mountain Chain as well as the East Antarctic Craton 
(Landsat Image Mosaic of Antarctica \url{http://lima.nasa.gov/pdf/A3_overview.pdf} (modified)).}
\label{ant-map}
\end{center}
\end{figure}



% \section{Introduction to Inversion Methods}

% see Liu Gu 