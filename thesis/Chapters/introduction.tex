%*******************************************************
% Introduction
%*******************************************************
% \pdfbookmark[0]{Introduction}{introduction}

\lhead{\emph{Introduction}}

\chapter{Introduction}

\section{Geology of Antarctica}
Antarctica is an appropriate choice for a seismic tomography study as it is the least seismically explored continent 
due to its remoteness and being surrounded by large oceans. 
Its lithosphere features some one-of-a-kind geological properties. 

Tectonically Antarctica is divided in two parts by the Transantarctic Mountains (TAM) spanning from the Ross Sea to the Weddel Sea.
The two sections are the West Antarctic Rift System (WARS) and the much larger and geologically older East Antarctic (EA) craton that 
was part of the Gondwana supercontinent \citep{Gupta2009}. 
% more on WARS (one of the largest continental rift zones) , or TAM not caused by contraction or subduction, ... from Morelli and references therein
The highest mountains in the TAM are up to 4500$\,$m high and the mountain chain is circa 3500$\,$km long \citep{Morelli2004}.
Locations of the two parts and the TAM can be seen in \autoref{ant-map}.
% origin of TAM 
The orogenesis % is this term here correct?
of the TAM is disputed, possible reasons are combinations of thermal uplift due to its proximity of the WARS continental rift, 
isostatic rebound after unloading of ice masses, flexural uplift after the breakup of two tectonic plates or 
the remains of a larger high plateau \citep{VanWijk2008}.
Model results from \citealp{VanWijk2008} suggest that the most likely reason for the uplift of the TAM is an extension along the 
boundary between the EA and Western Australia. This could also explain the origination of the nearby Wilkes depression.
Furthermore hotspots have be suggested to exist in Antarctica like under the active Erebus volcano \citep{Gupta2009}.

The size of the continent (larger than Europe) makes long period tomography suitable.  
Seismological tomography of Antarctica started with the construction of global models of wave velocity like \citealp{Woodhouse1984}.
It is known from previous tomographic analysis that the older, eastern part of the continent consists of seismically faster material than 
the younger Rift System in the West \citep{Morelli2004}.

Seismic research on Antarctica was for a long time limited by the few available seismometers at research stations on the continent. 
The worldwide installation of the GEOSCOPE network let to first maps of Antarctica's lateral heterogeneities \citep{Roult1994}.
Before 2007 seismic studies relied on global networks or temporarily installed seismometer networks, the most notable
example is the TAMSEIS experiment \citep{Lawrence2006}. 
After 2007 permanent broadband stations were installed within the Gamburtsev Antarctic Mountains Seismic Experiment (GAM-SEIS/AGAP)
as well as the Polar Earth Observing Network (POLENET ANET); 
in total there are now 77 year-round seismic stations available \citep{Anthony2014}.  


\section{Objective of this Thesis} % thesis in capital letters?

The objective of this work is to perform the initial steps needed for the improvement of a continental scale tomographic model for the crustal and upper mantle structure.
% For this a starting model like S20RTS within the Comprehensive Earth Model is used.
% by means of full seismic waveform inversion using the adjoint technique. 
Long period seismic signals of 60 to 120 seconds % check the numbers again
are used for the first iterations of the forward model. 
The limitation to only low frequencies and %number of iterations
iterations is due to the restricted time for the Master's thesis.
% The model updates are done by simulating a forward model and with its result an adjoint model. % improve this sentence
Testing the Large Scale Seismic Inversion Framework (LASIF) in general and with respect to its compatibility with SPECFEM3D 
GLOBE is a further intention of this work.
% The preprocessing and preparation for the adjoint simulations is performed with LASIF. 
The measured waveforms are dominated by surface waves. 
% are used for the tomography as 
They are more suitable for large upper mantle structures than body waves, as they are normally sensitive to a depth of a few 
hundred kilometers \citep{Morelli2004}.

% maybe more on concepts of inversion and tomography
 
% Why FWI was chosen
% simple classical ray method -> finite frequency methods -> FWI (fully numerical)


\begin{figure}[H]
\begin{center}
\includegraphics[width = 0.85\textwidth]{images/antarctica-map_edit.png}
\caption{This map shows the West Antarctic Rift System the Transantarctic Mountain Chain as well as the East Antarctic Craton 
(Landsat Image Mosaic of Antarctica \url{http://lima.nasa.gov/pdf/A3_overview.pdf} (modified)).}
\label{ant-map}
\end{center}
\end{figure}