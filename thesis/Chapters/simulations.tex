%*******************************************************
% Simulations
%*******************************************************
% \pdfbookmark[0]{Simulations}{simulations}
\lhead{\emph{Forward simulations}}

\chapter{The Spectral Element Method and Adjoint Simulations}

% The text about the SEM could also be part of the beginning 

\section{The Spectral Element Method}

\subsection{Why Spectral Elements}

Many different concepts have been used in the last decades for seismological simulations. 
While the Finite Difference (FD) Method is the easiest in terms of implementation, it is not suitable for 
complex geometries due to its regular grid. % (This should probably be explained better, problems with cubic elements and curved surfaces ...
Furthermore numerical dispersion errors limit the accuracy of this method.
The Pseudospectral Method offers much better accuracy, however the required global communications do not allow the 
implementation on parallel hardware. Without parallelisation large simulations like in this work are downright impossible.
This leaves then to choose between the Discontinuous Galerkin Method and the Spectral Element Method. As the latter one 
is already well established in the seismological community and the adjoint inversions have been performed before with the
SPECFEM software, I choose the SEM for this work. 
% DG harder to implement, SEM methods bit faster to run, should SeisSol be mentioned here? 
% This sentence here or in the next section? Also maybe with explanations for each advantage
The Spectral Element Method is characterised by its geometrical flexibility, small numerical errors (exact in the space domain), 
efficiency on parallel hardware and implicit free surface boundary conditions. 
The latter point makes this method especially suitable for surface wave simulation. % is this relevant? 

\subsection{Basic Concepts of the Spectral Element Method}

% How should Heiner's book be cited?

The equation that has to be solved with the SEM is the elastic wave equation. The one dimensional case is: 

\begin{equation}\label{1Dwave}
\rho(x) \ddot{u}(x,t) -  \partial_x \left[ \mu(x) \partial_x u(x,t) \right] = f(x,t)
\end{equation}

Where $\rho$ is the density, $u$ the unknown function (displacement), $\mu$ the shear modulus and $f$ the source or external forces.
%
The unknown function is approximated by superposing a finite amount of basis functions and the equation can then be written in matrix 
notation as follows: 

\begin{equation}\label{matrix_equation}
\underbrace{\textbf{M}}_{\text{Mass matrix}} \cdot \ \ddot{\textbf {u}} (t)  \ + \underbrace{\textbf{K}}_{\text{Stiffness matrix}} \cdot \ \textbf{u} (t) \, = \, \underbrace{\textbf{f} (t)}_{\text{Source term}} 
\end{equation}

%the mass matrix is defined as 
%\begin{equation} \label{mass}
%M_{ji} \ = \ \int_{G} \rho(x) \ \psi_j(x) \ \psi_i(x)  \ \mathrm{d}x
%\end{equation}
%
%the stiffness matrix is defined as 
%\begin{equation} \label{stiffness}
%K_{ji} \ = \ \int_{G} \mu(x) \ \partial_x \psi_j(x) \ \partial_x \psi_i(x) \ \mathrm{d}x
%\end{equation}

This equation is solved for every element in the domain. 
As the solution has to be found numerically the problem has to be discretised. 
The whole domain is therefore divided in mesh elements, which reflect the topography and internal discontinuities.
The elements can have different mechanical properties so that complex heterogeneous media can be modelled. 
Each element contains then an equal number of unevenly spaced (denser at the boundaries) grid points, the 
Gauss-Lobatto-Legendre (GLL) collocation points.
The interpolation with Lagrange polynomials as basis functions is exact at these grid points. 
%For each element in the domain the unknown function $u$ is exactly interpolated with Lagrange polynomials at unevenly spaced 
%(denser at the boundaries) grid points, the Gauss-Lobatto-Legendre (GLL) collocation points. 
The order of the Lagrange polynomials determines the number of collocation points per element and subsequently the accuracy of the simulation. 
%
SEM uses stress free boundary conditions and is therefore well suited for the simulation of the stress free state at the Earth's surface. 
%
As also the numerical integration is performed by GLL quadrature with the use of Lagrange basis functions, 
the orthogonality of these polynomials leads to a diagonal mass matrix, which makes explicit extrapolation possible. 
This is the main reason for the efficiency of the Spectral Element Method.

The elemental solutions that overlap at their boundaries are combined in a global system. 
This global set of equations is extrapolated with a Finite Difference scheme.
Numerical errors of the FD method together with inaccuracies of the numerical integration determine the total error of the SEM.



\section{Forward and Adjoint simulations}


\subsection{Overview}
% reality vs. model
To improve the Earth model a measure is required how close the simulation is to the real world. 
% phase vs. amplitude
There are two possibilities to quantify the differences between the observed seismograms and the synthetic data.
One can either measure the traveltime phase differences or consider amplitude differences.
While the traveltime is quasi-linearly linked to structural heterogeneities (which we are interested in),
the measured amplitude is effected nonlinearly by them and also strongly dependent on the local geology 
around the seismometer \citep{Fichtner2008}.
For this reason I focus on phase differences in this work.
% short intro to FWI
Owing to the application of the full waveform inversion (FWI) no specific seismic phases have to be identified 
individually \citep{Fichtner2011a}. 
% Furthermore the analyses for body and surface wave are naturally combined for the inversion. % Is that correct this way? See Fichtner2008 page 2
This would be required with classic tomography based on ray theory or finite frequency tomography that already 
accounts for 3D effects of wave propagation. % correct that way?
Whereas the two methods mentioned before can be solved (semi-)analytically, FWI is entirely based on numerical
solutions of the equations of motion.

\begin{figure}[h]
\begin{center}
% Probably better image requird with East component
\includegraphics[width = 1\textwidth]{images/fwi_scheme_2.png}
\caption{The general concept of full waveform inversion with the adjoint method}
\label{adjoint_scheme}
\end{center}
\end{figure}

The general procedure of FWI using the adjoint method is visualized in \autoref{adjoint_scheme}.
For every event and receiver pair synthetic waveforms are generated from the initial velocity model and 
compared with the observed ones.
%
% minimisation of misfit function
The goal is now to minimise the phase misfit function, which can be achieved by setting its gradient to zero.
% adjoint method
The computation of the gradient can be achieved using the Adjoint Method. 
For this technique the residuals between the forward simulation and the observed data are used as source time function 
for the adjoint simulation. 
The adjoint simulation starts at the receivers and propagates backward in time towards the actual earthquake source.
Multiple adjoint sources are modelled during one simulation, but just one earthquake location is used as receiver.
From the interaction between the forward and the adjoint simulation the gradient can be obtained. 
% misfit kernel
Summing up the gradients for all events leads to the misfit kernel. 
This kernel provides information about the sensitivity to perturbations in the model \citep{Magnoni2012}.
With these results the Earth model can be updated and new simulations can be started to iteratively further improve the model.

% probabilistic vs. deterministic approach, latter one used in this work, see Fichtner book 
In this thesis I try to identify the model parameters as good as possible (deterministic approach), instead of assigning
different probabilities to various possible parameter values as it is done with the probabilistic approach.

\subsection{Theory of the Adjoint Method}

Here the adjoint method is described in more detail.

The forward simulations of the events are performed with SPECFEM3D GLOBE, which is described in more detail in the next section.
%
%adapt paragraph to section above according to Fichtner 2008
%MISFIT
With the results of the forward model the misfit can be computed.
The phase misfit used with the LASIF Framework follows \citealp{Fichtner2009}:
%
% see page 4 in Fichtner 2008
\begin{equation} 
E_p^n(u_i^0, u_i) \, := \, \int_{\mathbb{R}^2} W_p^2(t,\omega) [\phi_i(t,\omega) - \phi_i^0(t,\omega)]^2 
   \, \mathrm{d}t \, \mathrm{d}\omega
\end{equation}
%
With $W_p$ symbolizing a positive weighting function, $\phi_i^0$ the observed phase and $\phi_i$ the synthetic phase.
The difference $\Delta \phi_i = \phi_i - \phi_i^0$ corresponds to a time shift $\Delta t$. 
And the phase difference can then be expressed as $\Delta \phi_i = \omega \Delta t = 2 \pi f \Delta t$.
Due to ambiguity this quantity cannot be interpreted if it is larger than $\pi$ (a half period).
The misfit function is defined as an $L_2$ norm. % L2 norm as in Fichtner 2009, Fichtner 2008 does have the L_n norm

The synthetic and observed data is Fourier transformed into the time-frequency domain with the Gaussian window
$h(t) = (\pi \sigma)^{\frac{1}{4}} e^{\frac{-t^2}{2 \sigma^2}}$
and the window parameter $\sigma$, which is close to the dominant frequency of the waveform:

\begin{equation}
\tilde{u}^0 (t, \omega) = F_h[u_i^0](t,\omega) := \frac{1}{\sqrt{2\pi}} \int_{-\infty}^{\infty} u_i^0 h(\tau - t) e^{-i \omega \tau} \mathrm{d}\tau
\end{equation}


% Weighting function
As small disturbances in the wavefield can cause large variations in the phase especially for small amplitudes
a weighting function $W_p$ is required.
%
%Furthermore the purposes of a weighting function are to avoid phase discontinuities, select certain time intervals of
%a seismograms or weighting them higher or lower and to reduce noise.
A weighting function is constructed as the product of Gaussian windows in time ($W_T$) and frequency ($W_F$) 
and a logarithm of the displacement function to emphasize waveforms with small amplitudes.
The temporal window represents the section of the synthetic seismogram that should be compared with the 
observed data. 
The frequency window helps to reduce noise and the depth resolution can be improved by weighing large frequencies up.
See \autoref{weight_func} for an example of how the weigthing function is applied to compute the adjoint sources.
Following \citealp{Fichtner2009} the phase weighting function has the following structure:

\begin{equation}
W_p(t, \omega) = W_T(t) W_F(\omega) \frac{  \log[1 + \vert \tilde{u}^0 (t, \omega) \vert ] }
{\max_{t,\omega} \log  [1 + \vert \tilde{u}^0 (t, \omega) \vert ]}
\end{equation}


%To avoid discontinuities at different frequencies with the phase difference a correlation function between
%the synthetic and observed frequencies and the Fourier transform of this function has to be computed; 
%this is described in more detail in \citealp{Fichtner2008}. % in more detail?
%However this procedure does not help if observations and simulation results are out phase an appropriate
%weighting function or suitable filter has to be applied. % Ist mir nicht so ganz klar

% the weighted phase change has to be bounded by $C||u||_2$ with $C<\infty$ \citealp{Fichtner2008}
% Approximation of $\Delta \phi$ by linear term of the Tayler series
% $W_p = |u~|$ so that phases with large amplitudes and frequencies close to the dominant period are emphasized
% $W_p$ as filter in the Time-Frequency domain


\begin{figure}[h]
\begin{center}
% Probably better image requird with East component
\includegraphics[width = 1\textwidth]{images/TF_domain_weight_func.png}
\caption[Misfit function and weighting. Graphic from \citealp{Fichtner2011}.]{The differences between the 
observed and synthetic seismograms are multiplied with a weighting function,
as some waves are too fast and some are too slow.
This results in adjoint sources for the backward simulations.
%Central plots: Weighted phase differences.
%Right plots: The used weighting functions.
%time-frequency domain and the weighting function $W_p$. 
Graphic from \citealp{Fichtner2011}(modified).}
\label{weight_func}
\end{center}
\end{figure}

% siehe Magnoni seite 58

% noch veraltet

% Übergang zwischen den Sätzen
%The misfit function takes the least squares difference between the synthetics $\boldsymbol{s}$ and the 
%observed data $\boldsymbol{d}$ for the selected time window from $0$ to $\boldsymbol{T}$ for each of the three components. 
%(The variable naming follows the doctoral thesis from \citealp{Magnoni2012}).
%
%\begin{equation}
%F(m) \ = \ \frac{1}{2} \sum_{r=1}^N \int_0^T \lVert  s(x_r, t, m) - d(x_r, t)  \lVert^2  \mathrm{d}t
%\end{equation}
%
%The location of the $N$ receivers is expressed by $x_r$. $m$ is the currently used model and $t$ is the time instance in the window.



% Variation of misfit function delta F
% simplification with Born

%Variations in the misfit function are due to perturbations in the displacement field. 
%These can be linearised with the Born approximation \citep{Liu2012}.
%This approximation %is a first order approximation and 
%considers only one-time scattered wave paths; multiple scattering is ignored.

% Instead of the Born approximation and the reciprocity of the Green's function the adjoint problem can also be solved with 
% Lagrange multiplier approach (\citealp{Liu2012} and references therein).

% reciprocity of Green's tensor

For the adjoint simulation the waveform adjoint source is defined as the sum of the residuals between observed and 
synthetic data for all receivers $N$ evaluated at the receiver position $x_r$ with a delta function 
% this sentence has to be changed to make sense

\begin{equation}
f_i^{\dagger}(x,t) \ = \ \sum_{r=1}^N [ s_i(x_r, T-t) - d_i(x_r, T-t) ] \delta (x-x_r)
\end{equation}

It is computed with LASIF and is used as source for the adjoint simulations.
The adjoint simulations are then started with multiple adjoint sources at all stations, where observed data for a 
specific event is available. 
For adjoint simulations seismograms are only obtained at the original earthquake location.

% Formula of the adjoint source here
%
% With the residuals between observed and synthetic data used as virtual sources at the receivers the adjoint simulation can be started.
%
% The results of the adjoint simulations make it possible to rewrite the variation of the misfit function.
With the saved displacement wavefields from the forward simulation the interaction of both wavefields can be computed as
convolution of both fields. 
The interaction of the two wavefields is then integrated over time to acquire the event kernel (Fr\'{e}chet derivative)
for density and P- and S-wave values. 
The event kernel is defined here as all kernels between one event location and all stations with adjoint sources for 
this event.
%The waveform misfit kernels (Fr\'{e}chet derivatives) can be computed with the results from the forward and the adjoint simulation. 
This way of computing the gradient of the misfit function is more efficient then for example with a scattering integral
approach \citep{Fichtner2006a}.
%
% misift kernel 
Summing up the event kernels gives the misfit kernel for all events.
The misfit kernel provides information about the sensitivity to perturbations in the model \citep{Magnoni2012}
and the negative gradient (misfit kernel) determines the direction to which the initial velocity model has to be modified. 
Smoothing has to be applied before the misfit kernel can be used for updates of the Earth model.
Optionally the misfit kernel can be weighted to emphasize certain depths or regions in the domain;
for example if a lot of stations are concentrated at one place like New Zealand in my data set.
By how much the velocity model is adapted in the first iteration is determined with a steepest descent line
search for the optimal step length.
Conjugate Gradient Optimisation could be applied if more than one iteration is available.

To compute the model update $\delta \boldsymbol{m}$ a preconditioner is applied to avoid local minima, which are most
often found next to the event or receiver location due to small scale heterogeneities there \citep{Fichtner2009}.
Therefore around these spots the preconditioner reduces the sensitivity.
For the calculation of the preconditioner the Hessian kernel is required. 
The Hessian kernel is defined as the second derivative of the misfit function. 
However it is not feasible to compute the exact Hessian kernel for large 3D simulation and therefore
it is approximated with only first order terms.

% For the following reasons the steepest descent method is applied in this work 
% 10. Juli; Lion: conjugate gradient kann erst ab der 2. Iteration angewendet werden
% daher verwenden wir steepest descent wie in F. Magnoni beschrieben

% see Fichtner book for conjugate gradient method (page 128)


\section{Simulating with SPECFEM and Application of the Adjoint Method}

\subsection{Introduction to SPECFEM}

SPECFEM was started by Dimitri Komatitsch and Jean-Pierre Vilotte in Paris at the 
Institut de Physique du Globe (IPGP) \citep{Vilotte1998} in the nineties and later expanded by Jeroen Tromp and many others. 
It is used for seismic wave propagation simulations on global and regional scale and it is one of the 
most used community softwares for this purpose. 
It is mostly written in Fortran2003 and different versions are available for 2D or 3D use cases as well
as for regional or global simulation. 
In this work the version SPECFEM 3D GLOBE 7.0 is used.
The speciality of this version is its cubed sphere approach \autoref{cube}.
This means a cube consisting of smaller cubes is distorted into the shape of a sphere.
The advantage of this approach is that wave propagation in regular hexahedrons is easier to compute than
for example in variably shaped tetrahedrons. % This sentence should be revised
The whole globe is divided into 6 chunks that each consist of smaller slices.

%partionining of mesh and distribution with MPI (Message Passing Interface) on several CPUs 


\begin{figure}[h]
\begin{center}
\includegraphics[width = 0.6\textwidth]{images/cubed_sphere_large.png}
\caption[The cubed sphere method.]{A large cube is built of smaller ones and then distorted into a sphere.
Graphic from the SPECFEM manual \citep{specfem_manual}.}
\label{cube}
\end{center}
\end{figure}


\subsection{Mesh Creation}

SPECFEM3D Globe comes with its own mesher, where the extent of the regional chunk of interest can be 
specified in a parameter file together with the number of slices in longitudinal and latitudinal direction.
The slices are required for the partitioning of the mesh, which is automatically done with the internal mesher.
It has to be checked if the created mesh in SPECFEM conforms well enough with the LASIF domain, 
otherwise the mesh has to be rotated around its central axis.
For this work the mesh and the domain corresponded well enough so that no rotations were required.
The timestep for the simulations is set to 0.14$\,$seconds.  
One of the six chunks of the globe is used for this regional mesh.
This chunk is subdivided into 16 slices along each edge, so 256 slices in total.
The number of slices determines on how many processors the mesh is created in parallel; 
it is the same number later used for the simulations. 
Every slice again contains smaller elements; the total number of elements in the mesh 
is 150,856. The Earth's inner core is not totally included, it would not play any role
for the simulations of this work anyway.

\begin{figure}[h]
\begin{center}
\includegraphics[width = 1\textwidth]{images/mesh_horizontal.png}
\caption[Overview of the mesh.]{The created mesh from two perspectives. The 256 slices are visible on the surface.
The colors indicate the different boundary layers of the Earth. The inner core is only partially
included. }
\label{mesh}
\end{center}
\end{figure}


\subsection{Initial Earth model}

As initial model I take the velocity data from the data-comprehensive seismic Earth model (CSEM) \citep{Afanasiev2014}.
The tetrahedral mesh accounts for topography, bathymetry and small heterogeneities due to its variable resolution.
It is constructed as long-wavelength global three dimensional model and also contains well resolved regional models
like sedimentary basins or slab structures, where they are available.
The CSEM \autoref{CEM} is planned to be updated with tomographic models from research groups worldwide.
Various seismic solvers can use CSEM as initial model and it is suitable for traveltime ray tomography as well as
full waveform inversion.

A subset of the CSEM model is created with the same dimension as my SPECFEM mesh. 
The normalized differences of the CSEM velocities to the one-dimensional ak135 model \citep{Kennett1995} are shown for three
depths in \autoref{CEM}.

\begin{figure}[h]
\begin{center}
\includegraphics[width = 1\textwidth]{images/CEM_visualisation/vsh_overview_equal_colorbar.png}
%\caption{The CSEM model as used for my work. Vertically polarized P-wave velocity (top left), 
%vertically polarized S-wave velocity (top right), horizontally polarized P-wave velocity (bottom left),
%horizontally polarized S-wave velocity (bottom middle), density (bottom right).}
\caption[Differences of CSEM to ak135 (same colorbar).]{The normalized differences of the CSEM model 
\citep{Afanasiev2014} compared to the 1D ak135 model. 
Here the horizontally polarized S-wave velocity $v_{sh}$ are displayed. Red color indicates that the ak135
velocities are smaller than the CSEM ones and blue symbolizes faster velocities.
As the same colorbar (dimensionless values) is used for all three models it can be clearly seen that the differences
to the 1D model get smaller with greater depth. }  
% Colorbar!!
\label{CEM}
\end{center}
\end{figure}

\begin{figure}[h]
\begin{center}
\includegraphics[width = 1\textwidth]{images/CEM_visualisation/vsh_overview_different_colorbar.png}
%\caption{The CSEM model as used for my work. Vertically polarized P-wave velocity (top left), 
%vertically polarized S-wave velocity (top right), horizontally polarized P-wave velocity (bottom left),
%horizontally polarized S-wave velocity (bottom middle), density (bottom right).}
\caption[Differences of CSEM to ak135 (individual colorbar).]{Here an individual colorbar is used for 
every depth layer, so that the differences to the 
ak135 model can be seen more clearly. 
The normalized differences of the CSEM model \citep{Afanasiev2014} compared to the 1D ak135 model. 
Here the horizontally polarized S-wave velocity $v_{sh}$ are displayed. Red color indicates that the CSEM
velocities are smaller than the ak135 ones and blue symbolizes faster velocities.}  
% Colorbar!!
\label{CEM}
\end{center}
\end{figure}


\subsection{Preparation and Workflow for the Simulations}
% All simulations are run on SuperMUC, one of the World's fastest supercomputers at the Leibniz-Rechenzentrum in Garching. 
% More on SuperMUC ? 


% Preparation for multiple simulations
% it was not described in the Manual
% directory structure with script 

Two simulation runs are required for the computation of the traveltime sensitivity kernel.
%
At first the forward simulations are performed for all pairs of earthquake and receivers, for which observed seismograms are available.
The simulations are run in parallel on the parallel supercomputer of the Leibniz-Rechenzentrum SuperMUC for all event-receiver pairs. 
Scripts to prepare the required directory structure are attached to this thesis. % Where ??????  CD, link 
One simulation is required for every event and seismograms are recorded at all stations. 
The state of the displacement, velocity and acceleration of the forward wavefield are saved for the later adjoint simulation.

% Window selection
In the next step an appropriate window, where observed and synthetic data agree well enough to be physically comparable,
is cut from the observed displacement seismograms and with the according processed section from the synthetic seismograms 
the adjoint sources are computed with LASIF.
The time window can be automatically selected with LASIF or alternatively with the misfit GUI \autoref{misfit_gui}.
% Therefore the parameters for the selection like the Signal to Noise Ratio are adapted.
%
A good procedure is to auto-select the windows and then check them manually in the LASIF GUI and 
add more windows if applicable.
For the event-receiver pairs with enough data this is done for all three components (North, East, Up).

\begin{figure}[h]
\begin{center}
% Probably better image requird with East component
\includegraphics[width = 1\textwidth]{images/misfit_gui.png}
\caption[The LASIF Misfit GUI.]{Windows can be selected automatically or manually with the LASIF Misfit GUI.
Sometimes no observed data is available like in this case for the northern component
or the data is too noisy too select a window like for parts of the eastern and vertical component in this example.}
\label{misfit_gui}
\end{center}
\end{figure}

% Calculation of Adjoint sources
% What part of the adjoint source computation is done with LASIF or SPECFEM?
With the prepared sources and the last state from the forward simulations the adjoint simulation can be started.
For all event-receiver pairs the synthetic earthquake starts now at the position of the receiver and propagates
backward in time.  
The forward simulation is back reconstructed from the saved last wavefield state. 
Its interaction with the adjoint simulation, which is propagating backwards in time,
yield the sensitivity kernels for the specific event. 
For every event the kernels for density $\rho$ as well as the $\alpha$ kernel for the P-wave velocity and the
$\beta$ kernel for the S-wave velocities are obtained. 
An example is given in the Results section \autoref{w_macq_kernel}.
Therefore SPECFEM3D convolves the adjoint with the forward wavefield.
The agreement between the adjoint displacement seismograms and the time-reversed forward seismograms should be very well. % testing
Differences can stem from numerical inaccuracies of the software. % correct?

The next step is to sum all 50 event kernels to one misfit kernel with the SPECFEM utilities.
Smoothing is then applied to the misfit kernels to remove small perturbations that are not based on physical phenomena as the 
resolution of the kernels is limited.
250$\,$km are used for the horizontal smoothing radius and 5$\,$km for the vertical one; the same values that were used for the
CSEM model.

The forward simulation for the 50 events cost 2544 CPU hours (ca. 106 days) and as the forward wavefields are saved the whole 
SPECFEM directory with all seismograms in the SAC Data File Format is larger than one Terabyte.
Only the synthetic seismograms for all events are 136 Gigabyte large.



% boundary kernels? 
% anisotropy? 

% summing of gradients for all event-receiver pairs

% update of Earth model

% iterations, which were not possible due to time constraints