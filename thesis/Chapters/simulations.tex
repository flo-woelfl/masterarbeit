%*******************************************************
% Simulations
%*******************************************************
% \pdfbookmark[0]{Simulations}{simulations}
\lhead{\emph{Forward simulations}}

\chapter{Simulations}

% The text about the SEM could also be part of the beginning 

\section{Why Spectral Elements}

Many different concepts have been used in the last decades for seismological simulations. 
While the Finite Difference (FD) Method is the easiest in terms of implementation, it is not suitable for 
complex geometries due to its regular grid. % (This should probably be explained better, problems with cubic elements and curved surfaces ...
Furthermore numerical dispersion errors limit the accuracy of this method.
The Pseudospectral Method offers much better accuracy, however the required global communications do not allow the 
implementation on parallel hardware. Without parallelisation large simulations like in this work are downright impossible.
This leaves then to choose between the Discontinuous Galerkin Method and the Spectral Element Method. As the latter one 
is already well established in the seismological community and the adjoint inversions have been performed before with the
SPECFEM software, I choose the SEM for this work. 
% DG harder to implement, SEM methods bit faster to run, should SeisSol be mentioned here? 
% This sentence here or in the next section? Also maybe with explanations for each advantage
The Spectral Element Method is characterised by its geometrical flexibility, small numerical errors (exact in the space domain), 
efficiency on parallel hardware and implicit free surface boundary conditions. 
The latter point makes this method especially suitable for surface wave simulation. % is this relevant? 

\section{Basic Concepts of the Spectral Element Method}

% How should Heiner's book be cited?

The equation that has to be solved with the SEM is the elastic wave equation. The one dimensional case is: 

\begin{equation}\label{1Dwave}
\rho(x) \ddot{u}(x,t) -  \partial_x \left[ \mu(x) \partial_x u(x,t) \right] = f(x,t)
\end{equation}

Where $\rho$ is the density, $u$ the unknown function (displacement), $\mu$ the shear modulus and $f$ the source or external forces.

The unknown function is approximated by superposing a finite amount of basis functions and the equation can then be written in matrix 
notation as follows: 

\begin{equation}\label{matrix_equation}
\underbrace{\textbf{M}}_{\text{Mass matrix}} \cdot \ \ddot{\textbf {u}} (t)  \ + \underbrace{\textbf{K}}_{\text{Stiffness matrix}} \cdot \ \textbf{u} (t) \, = \, \underbrace{\textbf{f} (t)}_{\text{Source term}} 
\end{equation}

%the mass matrix is defined as 
%\begin{equation} \label{mass}
%M_{ji} \ = \ \int_{G} \rho(x) \ \psi_j(x) \ \psi_i(x)  \ \mathrm{d}x
%\end{equation}
%
%the stiffness matrix is defined as 
%\begin{equation} \label{stiffness}
%K_{ji} \ = \ \int_{G} \mu(x) \ \partial_x \psi_j(x) \ \partial_x \psi_i(x) \ \mathrm{d}x
%\end{equation}

This equation is solved for every element in the domain. 
As the solution has to be found numerically the problem has to be discretised. 
The whole domain is therefore divided in mesh elements, which reflect the topography and internal discontinuities.
The elements can have different mechanical properties so that complex heterogeneous media can be modelled. 
Each element contains then an equal number of unevenly spaced (denser at the boundaries) grid points, the 
Gauss-Lobatto-Legendre (GLL) collocation points.
The interpolation with Lagrange polynomials as basis functions is exact at these grid points. 
%For each element in the domain the unknown function $u$ is exactly interpolated with Lagrange polynomials at unevenly spaced 
%(denser at the boundaries) grid points, the Gauss-Lobatto-Legendre (GLL) collocation points. 
The order of the Lagrange polynomials determines the number of collocation points per element and subsequently the accuracy of the simulation. 

Stress free boundary conditions are applied, which represent the stress free state at the Earth's surface. 

As also the numerical integration is performed by GLL quadrature with the use of Lagrange basis functions, the orthogonality of these polynomials 
leads to a diagonal mass matrix, which makes explicit extrapolation possible. This is the main reason for the efficiency of the Spectral Element Method.

The elemental solutions that overlap at their boundaries are then put together in a global system. 
This global set of equations is then extrapolated with a Finite Difference scheme.
Numerical errors of the FD method together with inaccuracies of the numerical integration determine the total error of the SEM.



\section{Introduction to SPECFEM}

SPECFEM \cite{Vilotte1998} was started by Dimitri Komatitsch and Jean-Pierre Vilotte in Paris at the Institut de Physique du Globe (IPGP) in the nineties 
and later expanded by Jeroen Tromp and many others. 
It is used for seismic wave propagation simulations on global and regional scale. 
The software is written in Fortran2003. \\
advantages ... \\
specifics: cubed sphere approach ... \\
own mesher, hexahedral elements (cubes) are used for SPECFEM \\
partionining of mesh and distribution with MPI (Message Passing Interface) on several CPUs 

In this work I use the Version SPECFEM3D GLOBE (Version 7.0). 


\section{Preparation for the Simulations}

All simulations are run on SuperMUC, one of the World's fastest supercomputers at the Leibniz-Rechenzentrum in Garching. 
The mesh is created with SPECFEM3D's own 3D meshing software. The mesh has to be put into the same position as the lasif 
domain by rotations around its central axis. % ????????? was rotation required now, is the description accurate enough


\chapter{Forward and Adjoint simulations}


\section{Overview}
To improve the Earth model a measure is required how close the simulation is to the real world. 
Then this misfit function has to be minimised by computing its gradient and setting it to zero. 
This can be achieved by using the Adjoint Method. For this technique the residuals between the forward simulation and the observed 
data is used as source time function % REALLY
for the adjoint simulation. 
The adjoint simulation starts at the receiver and propagates backward in time towards the actual earthquake source.
From the interaction between the forward and the adjoint simulation the gradient can be obtained. 
Summing up the gradients for all events and receivers leads to a misfit kernel. 
This kernel provides information about the sensitivity to perturbations in the model \citep{Magnoni2012}.
With these results the Earth model can be updated and new simulations can be started to iteratively further improve the model.


\section{The Adjoint Method}

The earthquake events are forward modelled with SPECFEM3D GLOBE. 
% Übergang zwischen den Sätzen
The misfit function takes the least squares difference between the synthetics $\boldsymbol{s}$ and the 
observed data $\boldsymbol{d}$ for the selected time window from $0$ to $\boldsymbol{T}$ for each of the three components. 
(The variable naming follows the doctoral thesis from \citealp{Magnoni2012}).

\begin{equation}
F(m) \ = \ \frac{1}{2} \sum_{r=1}^N \int_0^T \lVert  s(x_r, t, m) - d(x_r, t)  \lVert^2  \mathrm{d}t
\end{equation}

The location of the $N$ receivers is expressed by $x_r$. $m$ is the currently used model and $t$ is the time instance in the window.

% How exactly is the residual measured?

% Variation of misfit function delta F
% simplification with Born

Variations in the misfit function are due to perturbations in the displacement field. 
These can be linearised with the Born approximation \citep{Liu2012}.
This approximation %is a first order approximation and 
considers only one-time scattered wave paths; multiple scattering is ignored.

% Instead of the Born approximation and the reciprocity of the Green's function the adjoint problem can also be solved with 
% Lagrange multiplier approach (\citealp{Liu2012} and references therein).

% reciprocity of Green's tensor

For the adjoint simulation the waveform adjoint source is defined as the sum of the residuals between observed and 
synthetic data for all receivers $N$ evaluated at the receiver position $x_r$ with a delta function 
% this sentence has to be changed to make sense

\begin{equation}
f_i^{\dagger}(x,t) \ = \ \sum_{r=1}^N [ s_i(x_r, T-t) - d_i(x_r, T-t) ] \delta (x-x_r)
\end{equation}

It is computed with LASIF and is used to start the adjoint simulations.

% Formula of the adjoint source here

% With the residuals between observed and synthetic data used as virtual sources at the receivers the adjoint simulation can be started.

The results of the adjoint simulations make it possible to rewrite the variation of the misfit function.
The interaction between forward and adjoint simulation can be expressed as waveform misfit kernels (Fr\'{e}chet derivatives)
for density and elastic tensor variations 


The waveform misfit kernels (Fr\'{e}chet derivatives) can be computed with the results from the forward and the adjoint simulation. 

This way of computing the gradient of the misfit function is more efficient then for example with a finite difference 
approach \citep{Fichtner2006a}.

% minimisation of gradients 

 
The gradients from the adjoint simulations are then summed up to acquire event kernels. %????????

Summing up the event kernels gives a misfit kernel. The misfit kernel provides information about the sensitivity to perturbations in the model \citep{Magnoni2012}.

The Earth model can then be updated by applying steepest descent or conjugate gradient optimisation algorithms. 

% For the following reasons the steepest descent method is applied in this work 


\chapter{Earth model updates}

As initial model I take the velocity data from the Comprehensive Earth Model (CEM) \citep{Afanasiev2014}.

