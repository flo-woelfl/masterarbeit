%*******************************************************
% Abstract
%*******************************************************
% \pdfbookmark[1]{Abstract}{Abstract}
% \chapter*{Abstract}

In this Master thesis the first steps towards a continental scale seismic full waveform inversions 
using the adjoint method are performed. Additionally the Large Scale Seismic Inversion Framework 
(LASIF) is tested and optimized for inversions using the SPECFEM3D Globe solver.
The geographical region of interest for the inversion is Antarctica for several reasons:
(i) All other continents are already better studied with seismic tomographies.
(ii) In recent years more stations were installed on the continent so that the availability increased.
(iii) The coverage of Antarctica is quite well as the continent is surrounded by oceanic spreading zones 
and stations are installed on the surrounding continents.
(iv) The geology of Antarctica is not well known and a seimic tomography could give more information
for example on the orogenesis of the Transantarctic Mountains.

Waveforms are downloaded for 50 earthquakes and processed with LASIF tools. 
Synthetic waveforms are generated with SPECFEM3D and their misfit to the observed data is taken as
adjoint sources for the adjoint kernel simulations. 
The event kernels are then summed up to acquire the misfit kernel and smoothing is applied to the kernel.
In further research the here completed steps could be used to update the initial velocity model and 
proceed with more iterations.

% This thesis also serves as guide on how to do adjoint simulations with SPECFEM3D Globe with the 
% help of LASIF.
 




