%*******************************************************
% Data acquisition and processing
%*******************************************************
% \pdfbookmark[0]{Data Acquisition and Processing}{data_processing}

\chapter{Data Acquisition}

The domain is chosen so that enough stations are available from the continents that surround Antarctica.
It is centred at the South Pole and extends 115$^\circ$ in North-South direction and 120$^\circ$ in
East-West direction. 
For the selection of the earthquakes the built-in LASIF feature \
add\_gcmt\_events is used to select 50 events equally distributed in the
domain (see \autoref{eqmap}). Nearly six gigabyte of waveform data are acquired this way.
The earthquakes cover a range of Moment Magnitudes between 5.2 and 6.3 (\autoref{mag_dist}). 
With 48 out of 50 most earthquakes having a depth smaller than 50\,km. Among these most 
events are much shallower than 50\,km. Only two events are in a depth range of 150 to 200\,km. 
No larger events are chosen as the point source approximation would not hold otherwise. 
The events are relatively equally distributed between 2005 and 2014 (\autoref{temp_dist}). No events before 2005 can 
be downloaded by LASIF, which is not a problem as the instrumentation before then was very scarce on Antarctica. 
The coverage of the Antarctic continent is visualised with a raydensity plot (\autoref{raydens}) that connects all 
stations with the recorded earthquakes. It can clearly be seen that the highest aggregation of stations exists in 
New Zealand due to its position on a plate boundary. \\


\begin{figure}[H]
\begin{center}
\includegraphics[width = 0.85\textwidth]{images/eqmap.png}
\caption{The red lines indicate the borders of the domain. The size of the ellipsoids depend on the magnitude of each event.}
\label{eqmap}
\end{center}
\end{figure}

\begin{figure}[H]
\begin{center}
\includegraphics[width = 1\textwidth]{images/magnitude_hist.eps}
\caption{Moment magnitudes range from 5.2 to 6.3, while most events are smaller than 5.6.}
\label{mag_dist}
\end{center}
\end{figure}

\begin{figure}[H]
\begin{center}
\includegraphics[width = 1\textwidth]{images/temporal_distribution.png}
\caption{The earthquakes are equally spread over the interval from 2005 to 2014.}
\label{temp_dist}
\end{center}
\end{figure}

\begin{figure}[H]
\begin{center}
\includegraphics[width = 1\textwidth]{images/magnitude_vs_depth.png}
\caption{The earthquakes cluster between 10 and 20\,km, 2 are between 150 and 200\,km.}
\label{depth_scatter}
\end{center}
\end{figure}

\begin{figure}[H]
\begin{center}
\includegraphics[height = 0.95\textheight]{images/depth_histogram.eps}
\caption{The earthquakes cluster between 10 and 20\,km, 2 are between 150 and 200\,km.}
\label{depth_dist}
\end{center}
\end{figure}

\begin{figure}[H]
\begin{center}
\includegraphics[width = 1\textwidth]{images/raydensity.png}
\caption{The raydensity plot connects all stations (triangles) with the events (beachballs) for which waveforms exist.}
\label{raydens}
\end{center}
\end{figure}

As the station from the Global Seismographic Network (GSN) are the most important ones due to their high standards on used 
instruments and continuous recording, it is made sure that all available waveforms can be used in the LASIF project. 
Therefore data missing from the LASIF downloaded was acquired from IRIS via an obspy fdsn client. 
The waveform data is then horizontally correctly rotated and StationXML files are downloaded from the IRIS website to
ensure the availability of instrumentation information etc. for all required time intervals. \\

After the data is collected it is checked if station information is available for each waveform, if duplicates exist, if 
waveforms are corrupted and all raypaths from station to event are within the domain. 
If the last check fails an automatic script from LASIF is created to delete the violating raypaths. 
When all the checks pass the waveforms can be preprocessed. 


\chapter{Data Preprocessing}

The preprocessing starts with the creation of a new iteration for which the frequency band and the seismic solver is selected.
For all the iterations SPECFEM3D GLOBE CEM is selected as solver and the following frequency bands are chosen:

\begin{table}[H]
\begin{center}
\begin{tabular}{{l|ccc}}
Iteration number  &1    			&2 				&3 \\
\hline
Frequency band    &60-120\,seconds	&40-70\,seconds	&60-120\,seconds          \\
\end{tabular}
\end{center}
\end{table}

Hereby the larger value for the period indicated the corner frequency for the highpass filter and the lower value the 
corner frequency for the lowpass filter. Hence the preprocessing can be understood as the application of a bandpass filter. 
Filtering is required not just to remove noise, but also to make it comparable to the later created synthetic waveforms.
Besides filtering furthermore the instrument response is removed and the data is detrended and demeaned. 
