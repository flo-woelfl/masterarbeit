
\chapter{Results}

\section{Event kernels}

Isotropic event kernels are generated during the adjoint simulations from SPECFEM for density $\rho$, P-wave velocity ($\alpha$) 
and S-wave velocity ($\beta$). As example the beta kernels for one event southwest of Africa is shown in \autoref{sw_africa_beta_kernel},
\autoref{sw_africa_donut} and \autoref{sw_africa_banana}.

Smoothing is applied for every kernel to remove small scale effects that do not give information about physical 
properties of the Earth.
The effects of smoothing with 250$\,$km in horizontal and 5$\,$km in vertical direction can be seen
in \autoref{smoothing}.

\begin{figure}[h]
\begin{center}
\includegraphics[width = 1\textwidth]{images/kernels/sw_africa_smoothing_50km_beta_kernel_arrow.png}
\caption[Effects of smoothing on an event kernel]
{The $\beta$ event kernel at 50$\,$km depth of an event 
Southwest of Africa with magnitude 5.4 (2014-3-8).
The left side shows the not smoothed kernel and the
right side shows the kernel smoothed with 250$\,$km in 
horizontal and 5$\,$km in vertical direction.
The beachballs are shown to indicate the event location.
The shown surface is at a depth of 50$\,$km.}  
% Colorbar!!
\label{smoothing}
\end{center}
\end{figure}

% Write about depth dependent weighting 

%\begin{figure}[h]
%\begin{center}
%\includegraphics[width = 1\textwidth]{images/kernels/west_macquarie_event_kernel_100km.png}
%\caption{The $\alpha$ event kernel at 100$\,$km depth of an event West of the MacQuarie island with a moment magnitude of 6.1 (2006-11-16).}  
%% Colorbar!!
%\label{w_macq_kernel}
%\end{center}
%\end{figure}


%\begin{figure}[h]
%\begin{center}
%\includegraphics[width = 0.85\textwidth]{images/kernels/beta_kernel_southwest_of_africa_50km_depth_smooth_edit.png}
%\caption[Smoothed $\beta$ event kernel at 50$\,$km depth for one event]{The smoothed $\beta$ event kernel at 50$\,$km depth for an event 
%Southwest of Africa with magnitude 5.4 (2014-3-8). The beachball of the event is added as reference.}  
%% Colorbar!!
%\label{sw_africa_beta_kernel}
%\end{center}
%\end{figure}


\begin{figure}[h]
\begin{center}
\includegraphics[width = 1\textwidth]{images/kernels/beta_kernel_southwest_of_africa_donut_slice-90to90_smooth_depth_weight.png}
\caption[Slice from longitude -90$^\circ$ to 90$^\circ$ through the smooth $\beta$ event kernel for one event]{A slice from 
longitude -90$^\circ$ to 90$^\circ$ through the smooth $\beta$ event kernel for an event Southwest of Africa with 
magnitude 5.4 (2014-3-8). Depth dependent weighting is applied to enhance the structure of the kernel in greater depth.}  
\label{sw_africa_donut}
\includegraphics[width = 1\textwidth]{images/kernels/beta_kernel_southwest_of_africa_banana_slice_0to180_smooth_depth_weight.png}
\caption[Slice from longitude 0$^\circ$ to 180$^\circ$ through the smooth $\beta$ event kernel for one event]{A slice from longitude 
0$^\circ$ to 180$^\circ$ through the smooth $\beta$ event kernel for an event Southwest of Africa with magnitude 5.4 (2014-3-8).
 Depth dependent weighting is applied to enhance the structure of the kernel in greater depth.}  
\label{sw_africa_banana}
\end{center}
\end{figure}


\section{Misfit kernel}

The misfit kernel is the sum of all 50 event kernels. \autoref{sw_africa_s_india_msifit_beta_kernel} shows an example of a 
smoothed misfit kernel for two events (Southwest Africa and South Indian Ocean) in 50$\,$km depth.
Isosurfaces of the smooth misfit kernels can be visualized in 3D with ParaView, an example can be seen in \autoref{3d_misfit_sw_af}.

%\begin{figure}[h]
%\begin{center}
%\includegraphics[width = 0.85\textwidth]{images/kernels/smooth_misfit_kernel_sw_africa_s_india_50km.png}
%\caption[Surface of a smooth misfit kernel for two events]{A smoothed $\beta$ misfit kernel summed up from two events Southwest of Africa and the South Indian Ocean.
%The depth of the shown surface is 50$\,$km.}  
%% Colorbar!!
%\label{sw_africa_s_india_msifit_beta_kernel}
%\end{center}
%\end{figure}

\begin{figure}[h]
\begin{center}
\includegraphics[width = 0.85\textwidth]{images/kernels/misfit_kernel_sw_afr_s_indian_ocean.eps}
\caption[Surface of a smooth misfit kernel for two events]
{The smoothed $\beta$ kernels at a depth of 50$\,$km are shown for two events southwest of Africa and 
in the South Indian Ocean are shown. The Moment tensors are added for both events.
The addition of both kernels yields the misfit kernel (50$\,$km depth).
For visibility the colorbar range of the southwest of Africa event is ten times larger than for the event 
in the South Indian Ocean.
Therefore the event southwest of Africa clearly dominates the combined misfit kernel.}  
% Colorbar!!
\label{sw_africa_s_india_msifit_beta_kernel}
\end{center}
\end{figure}

%\begin{figure}[h]
%\begin{center}
%\includegraphics[width = 0.85\textwidth]{images/kernels/smooth_misift_sw_africa_s_india_val1e-7.png}
%\caption[3D isosurface of a misfit kernel for two events.]{Three dimensional isosurface of the smoothed $\beta$ 
%misfit kernel summed up from two events Southwest of Africa and the South Indian Ocean. 
%%The value of the isosurface is $1e-7 s/km^3$.
%}  
%% Colorbar!!
%\label{3d_misfit_sw_af}
%\end{center}
%\end{figure}


\begin{figure}[h]
\begin{center}
\includegraphics[width = 1\textwidth]{images/kernels/misfit_50_events_50km_depth.png}
\caption[Smooth misfit kernel for 50 events.]{Smooth misfit kernel added up from
all 50 events used in this work. The shown surface is 50$\,$km deep.}  
% Colorbar!!
\label{misfit_50}
\end{center}
\end{figure}



\begin{figure}[h]
\begin{center}
\includegraphics[width = 0.85\textwidth]{images/kernels/misfit_50events_3D.png}
\caption[3D isosurface of a misfit kernel for 50 events.]{Three dimensional isosurface of the smoothed $\beta$ 
misfit kernel summed up from all 50 events. Depth dependent weigthting is applied to enhance deeper structures.
%The value of the isosurface is $1e-7 s/km^3$.
}  
% Colorbar!!
\label{3d_misfit_50}
\end{center}
\end{figure}