
% Tutorial for the preparation, simulation and visulisation

\chapter{Tutorial for LASIF and SPECFEM3D GLOBE}
\lhead{\emph{Tutorial}}

\section{Data Acquisition and Preprocessing with LASIF}
To start a LASIF project and download data see the tutorial on \url{http://krischer.github.io/LASIF/tutorial.html}.
After all events are created in the LASIF directory, it is advisable to create scripts for the download of the data of the form.

\begin{lstlisting}
lasif download_data [event_1]
lasif download_data [event_2]
\end{lstlisting}

After the download the data should be validated to discover broken files or
delete raypaths that are not in the LASIF domain.

\begin{lstlisting} 
lasif validate_data --full
\end{lstlisting}

The data can be visualized with a raydensity plot:
\begin{lstlisting} 
lasif plot_raydensity --plot_stations
\end{lstlisting}

A new iteration for SPECFEM is created in the following way:
\begin{lstlisting} 
lasif create_new_iteration 1 [low period] [high period] SPECFEM3D_GLOBE_CEM
\end{lstlisting}

For my first iteration I use 60 seconds as lower period and 120 seconds as the
higher period. 1 refers to the iteration name in the upper example.
% Iteration length / time steps

The preprocessing of data should be run in parallel:
\begin{lstlisting} 
mpirun -n 2 lasif preprocess_data 1
\end{lstlisting}

If no event is specfied the preprocessing takes place for all events.
To test the validity of the processed waveforms they can be viewed with
the inline obspy-plot functionality.

Before the start of the simulations with SPECFEM3D GLOBE the CMT for every earthquake can be generated.
\begin{lstlisting} 
lasif  generate_input_files 1 [event_name]
\end{lstlisting}
1 again refers to the iteration name. 
Alternatively they can be downloaded from the Global Centroid-Moment-Tensor (CMT) Project
\url{http://www.globalcmt.org/CMTsearch.html}.


\section{SPECFEM3D GLOBE - Installation and Simulation}

This tutorial deals with the installation of SPECFEM3D GLOBE on the 
LRZ SuperMUC Cluster. 
First download the repository from
\begin{lstlisting} 
git clone --recursive --branch devel https://github.com/geodynamics/specfem3d_globe.git
\end{lstlisting}
tar it and scp it to SuperMUC.

SPECFEM is then configured in the following way:
\begin{lstlisting} 
module load hdf5/mpi
module load netcdf/mpi
./configure FC=ifort MPIFC=mpif90 -v --with-cem CEM_FCFLAGS="$HDF5_INC  $NETCDF_INC" CEM_LIBS="$HDF5_F90_LIB $NETCDF_F90_LIB"
\end{lstlisting}
for the use of the CSEM velocity models.

Before the first run the Par\_file has to be adapted  to match the 
dimensions of the lasif domain. 
The Par\_file for a forward and an adjoint simulation is added to this thesis.
After meshfem3D is made it can be executed. 
For every larger program execution the job has to be submitted over the SuperMUC
job scheduler with a specific job file with the command \texttt{llsubmit}.
In case anything went wrong with the job it can be cancelled with 
\texttt{llcancel} [job\_id] and currently running jobs can be seen with \texttt{llu}, which also provides the job\_id.
More information on this can be found here: \url{https://www.lrz.de/services/compute/supermuc/loadleveler/}.
The job file for the mesher is added on the supplements of this work.

To visualize the mesh and compare it with the LASIF domain the option 
\texttt{SAVE\_MESH\_FILES} is set to true. 
For the acquisition of the CSEM model the \texttt{MODEL} parameter is set
to \texttt{CEM\_REQUEST}. 
The created .xyz files can be sent to the CSEM group to get the CSEM files.
Then it is possible to create graphic files from the SPECFEM main directory:
\begin{lstlisting} 
make combine_AVS_DX
./bin/xcombine_AVS_DX
\end{lstlisting}
The generated mesh file can be found as \texttt{AVS\_fullmesh.inp}.
It can be viewed with ParaView.
See more on visualisation in chapter 8 of the SPECFEM Manual.
The mesh files are saved in \texttt{DATABASES\_MPI}. 
It is advisable to make a copy of this folder to reuse the mesh
for following simulations. 
The mesh has to be created only once.

As a next step the simulations can be prepared. 
The \texttt{CMTSOLUTION} generated with LASIF is put into the DATA directory.
As mentioned in the Conclusion there were memory problems with the 
computation of exact attenuation and therefore i set the following parameters:
\begin{lstlisting} 
PARTIAL_PHYS_DISPERSION_ONLY = .true.
UNDO_ATTENUATION = .false.
\end{lstlisting}

The \texttt{RECORD\_LENGTH\_IN\_MINUTES} has to be set to the desired length. 
I used 60 minutes. 
The velocity model is set to \texttt{CEM\_ACCEPT} after the CSEM values are 
put into the DATA/cemRequest folder.
The resulting seismograms are saved in the SAC Data File Format
with \texttt{OUTPUT\_SEISMOS\_ASCII\_TEXT = .true.}. 
No other formats are required for LASIF and would only need unnecessary 
space on SuperMUC.

For several simultaneous runs several directories have to be created.
The number of simultaneous runs is specfied in the Par\_file.
There is one main directory structure with \texttt{DATA}, \texttt{DATABASES\_MPI}
and \texttt{OUTPUT\_FILES} folder.
One large \texttt{CMTSOLUTION} with all events has to be put in the main 
\texttt{DATA} folder, it can be created with \texttt{cat [event\_1]/CMTSOLUTION}
\texttt{[event\_2]/CMTSOLUTION > DATA/CMTSOLUTION}.
It is important that all events are put one below the other with no 
blank spaces in between. 
The total number of lines in the file has to be a multiple of 13 
(number of lines per one CMT file).
Then there is a \texttt{run00xx} folder for every event.
To create this directory structure I created a small script \texttt{step\_1\_simultaneous\_runs\_directories.sh} that is added to 
the attached material. 
It should only be executed after the meshing is completed or the 
resulting mesh files are copied into the main SPECFEM directory.
The correct \texttt{CMTSOLUTION} for every event has to be put in the 
respective run directory in the same order the events are in the large
\texttt{CMTSOLUTION} file.
SPECFEM is then again executed with the job scheduler. 
The respective job\_file is attached.
The synthetic seismogram for every event are saved in the 
\texttt{run00xx/OUTPUT\_FILES} folder and can then be copied to the 
LASIF directory with rsync.



\section{Misfit and Adjoint Sources with LASIF}

The processing of the synthetic data happens as soon as they are opened
by any utility within LASIF like the Misfit GUI.
The misfits can be manually selected with \texttt{lasif launch\_misfit\_gui}.
Alternatively they can be automatically selected with:
\begin{lstlisting} 
lasif select_windows 1 [event_name]
\end{lstlisting}
The created windows can be viewed with Misfit GUI and more windows can be added
if appropriate.
The automatic selection can be customized in \\ \texttt{FUNCTIONS/window\_picking\_function.py}. 
The windows are saved in the \\
\texttt{ADJOINT\_SOURCES\_AND\_WINDOWS} directory
and now the adjoint sources can be created:
\begin{lstlisting} 
lasif finalize_adjoint_sources 1 [event_name]
\end{lstlisting}

\section{Adjoint Simulations}
The created adjoint sources as well as the file \texttt{STATIONS\_ADJOINT} 
are then copied to the SEM folder in the respective \texttt{runxx} folder 
on SuperMUC with rsync. 
The \texttt{STATIONS\_ADJOINT} file can be moved to the individual 
\texttt{DATA} folder of every event with\\ \texttt{step\_2\_move\_STATIONS\_ADJOINT.sh}.
To start the adjoint simulations the parameters have to be set like
\begin{lstlisting} 
SIMULATION_TYPE                 = 3        
NOISE_TOMOGRAPHY                = 0        
SAVE_FORWARD                    = .false.   
\end{lstlisting}

the option \texttt{APPROXIMATE\_HESS\_KL = .true.} is needed to compute 
preconditioners for the kernel.
The executable specfem3D has to be compiled again (\texttt{make clean}, \texttt{make specfem3D}).
The created adjoint seismograms can be compared with the forward synthetic
seismograms and should fit very well.


\section{Kernel Visualisation and Smoothing}

The kernel files are saved in the \texttt{run00xx/DATABASES\_MPI} directories
as \texttt{*beta\_kernel*} files. 
With the standard option in the parameter file the isotropic kernels
$\alpha$, $\beta$ and $\rho$ as well as bulk\_c are created.
The visualisation of the kernels is described in chapter 8 of the 
SPECFEM Manual. Here some advices on the required files for the visualisation.
A \texttt{slices\_minor} file is needed that can be created as follows
(starting from the SPECFEM main directory):
\begin{lstlisting} 
cd utils/Visualization/VTK_Paraview/
cp ../../../DATA/Par_file ../../../DATA/CMTSOLUTION ../../../DATA/STATIONS_ADJOINT .
./global_slice_number.pl CMTSOLUTION STATIONS_ADJOINT Par_file
\end{lstlisting}

This description is for one single event kernels, if simultaneous runs were used
the paths to \texttt{run00xx} folders have to be added. 
However the \texttt{slices\_minor} file is the same for all runs, so it has to be 
done only once.

As next step the kernel files are collected (again from the main directory):
\begin{lstlisting} 
cd utils/collect_database
./copy_m_globe_database.pl slice_minor lsf_machine_file beta_kernel [jobid]
\end{lstlisting}

The previously created \texttt{slice\_minor} file has to be copied in this
directory. The \texttt{lsf\_machine} simply lists the number of nodes and processors. In my case it is a one liner file of the form \texttt{4 256}.

For isotropic kernels a parameter has to be set to true in the file 
\texttt{setup/constants\_tomography.h} (line 44).
\begin{lstlisting}
! if you prefer isotropic  (alpha,beta,rho) kernels, set this flag to true
logical, parameter :: USE_ALPHA_BETA_RHO = .true.
\end{lstlisting}

The smoothing of the kernels is done with SPECFEM utilities
\begin{lstlisting} 
cd src/tomography/postprocess_sensitivity_kernels/
make
\end{lstlisting}

Again a job script is needed for the smoothing that is also attached to this work.
The essential line of the job file is this one:
\begin{lstlisting} 
mkdir OUTPUT_FILES/smooth_beta_kernel
mpiexec -np 512 ./bin/xsmooth_sem 250 5 beta_kernel DATABASES_MPI OUTPUT_FILES/smooth_beta_kernel
\end{lstlisting}
The required number of processors is the same as was used for the simulations.
For misfit kernels \texttt{OUTPUT\_SUM} is used instead of \texttt{DATABASES\_MPI}.

Now the kernels can be visualized:
\begin{lstlisting} 
make combine_vol_data_vtk
ln -s ../DATABASES_MPI/*solver_data.bin .
./bin/xcombine_vol_data_vtk slice_list beta_kernel_smooth OUTPUT_FILES/smooth_beta_kernel OUTPUT_FILES/smooth_beta_kernel OUTPUT_FILES/smooth_beta_kernel 0
\end{lstlisting}

The rough kernels can of course also be visualized if 
\texttt{beta\_kernel} is used instead of \texttt{beta\_kernel\_smooth}.
The \texttt{*solver\_data.bin} files are required for the visualisation.
As the $\beta$ kernel is only created for region 1 there comes a error message that 
it cannot be visualized for region 2 and 3. This is not a problem and does not occur
for the other kernels.

The general structure is given in the manual as:
\begin{lstlisting} 
./bin/xcombine_vol_data_vtk slice_list kernel_filename input_topo_dir input_file_dir output_dir low/high-resolution-flag-0-or-1 [region]
\end{lstlisting}

Here again only the example for a single run and the beta kernel are given. 
Alternatively \texttt{alpha\_kernel} , \texttt{bulk\_c\_kernel} and \texttt{rho\_kernel} can be visualised the same way.
The slice list is a file containing all slice numbers from 0 to the 
number of slices - 1 in one column. In my case from 0 to 255.
The resulting \texttt{*.vtk} files can be viewed in ParaView.
For the $\beta$ kernel only region 1 (crust and upper mantle) is created.
Spherical and plane slices can be applied and the color range adapted 
with ParaView tools.

\section{Misfit kernels}

The event kernels can then be added up to a misfit kernel.
It is best if all event kernels are part of one simulation with
several simultaneous runs.

\begin{lstlisting} 
cd src/tomography
make
cd ../..
mpiexec -np 512 ./bin/xsum_kernels kernels_list.txt INPUT_KERNELS OUTPUT_SUM
\end{lstlisting}

Again a job file is required here for parallel computing (in the attachment).
The \texttt{kernels\_list.txt} file contains the paths to the {run00xx/DATABASES\_MPI} folders from the \texttt{INPUT\_KERNELS}.
Symbolic links to the {run00xx/DATABASES\_MPI} have to be created in the
\texttt{INPUT\_KERNELS}.
The misfit kernels can then be smoothed and a \texttt{*.vtk} file can be 
generated as described above.
