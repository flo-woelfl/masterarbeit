% book example for classicthesis.sty
\documentclass[
  % Replace twoside with oneside if you are printing your thesis on a single side
  % of the paper, or for viewing on screen.
  %oneside,
  twoside,
  11pt, a4paper,
  footinclude=true,
  headinclude=true,
  cleardoublepage=empty
]{scrbook}

\usepackage{lipsum}
\usepackage[linedheaders,parts,pdfspacing]{classicthesis}
\usepackage{amsmath}
\usepackage{amsthm}
\usepackage{acronym}

\title{Tomography of Antarctica}
\author{Florian W\"olfl}

\begin{document}

\maketitle

%*******************************************************
% Abstract
%*******************************************************
\pdfbookmark[1]{Abstract}{Abstract}
\chapter*{Abstract}

Short summary of the contents of your thesis.

%*******************************************************
% Dedication
%*******************************************************
\thispagestyle{empty}
\pdfbookmark[1]{Dedication}{Dedication}

\vspace*{3cm}

\begin{center}
    To someone special 
\end{center}

%*******************************************************
% Acknowledgments
%*******************************************************
\pdfbookmark[1]{Acknowledgements}{acknowledgements}
\chapter*{Acknowledgements}



%*******************************************************
% Declaration
%*******************************************************
\pdfbookmark[0]{Declaration}{declaration}
\chapter*{Declaration}
\thispagestyle{empty}

Selbst{\"a}ndigkeitserkl{\"a}rung \\
Hiermit versichere ich, diese Masterarbeit selbst{\"a}ndig und lediglich unter Benutzung der angegebenen Quellen und Hilfsmittel verfasst zu haben. 
Ich erkl{\"a}re weiterhin, dass die vorliegende Arbeit noch nicht im Rahmen eines anderen Pr{\"u}fungsverfahrens eingereicht wurde.\\

\noindent Statement of authorship \\
With this statement I certify that this master thesis has been composed by myself. Unless otherwise acknowledged in the text, it describes my own work. All references have been quoted and all sources of information have been specifically acknowledged. This thesis has not been accepted in any previous application for a degree.  

%*******************************************************
% Table of Contents
%*******************************************************
\pdfbookmark[1]{\contentsname}{tableofcontents}

\setcounter{tocdepth}{2} % <-- 2 includes up to subsections in the ToC
\setcounter{secnumdepth}{3} % <-- 3 numbers up to subsubsections

\tableofcontents 

%*******************************************************
% List of Figures and of the Tables
%*******************************************************

%*******************************************************
% List of Figures
%*******************************************************    
\pdfbookmark[1]{\listfigurename}{lof}
\listoffigures

%*******************************************************
% List of Tables
%*******************************************************
\pdfbookmark[1]{\listtablename}{lot}
\listoftables
  
%*******************************************************
% List of Listings
%******************************************************* 
\pdfbookmark[1]{\lstlistlistingname}{lol}
\lstlistoflistings 
   
%*******************************************************
% Acronyms
%*******************************************************
\pdfbookmark[1]{Acronyms}{acronyms}
\chapter*{Acronyms}
\begin{acronym}[UML]
    \acro{CEM}{Comprehensive Earth Model}
    \acro{LASIF}{LArge Seismic Inversion Framework}
\end{acronym} 


\part{Introduction}

\begin{itemize}
\item What is inversion and tomography 
\item What is the adjoint method
\item Why Antarctica
\end{itemize}


\chapter{A Chapter}
\lipsum[1] % This is just some filler text. Remove before you start writing

\section{A Section}


\part{Data acquisition and processing}

For the selection of the earthquakes the built-in LASIF feature 
add\_gcmt\_events is used to select 50 events equally distributed in the
domain. The earthquakes cover a range of Moment Magnitudes between 5.2 and 6.3. 
No larger events were chosen as the point source approximation would not hold otherwise. 
The events are relatively equally distributed between 2005 and 2014. No events before 2005 can 
be downloaded by LASIF, which is not a problem as the instrumentation at before then was very scarce on Antarctica. 


\chapter{Yet Another Chapter}

  
\
\part{Simulations}
\chapter{ Chapter}
\lipsum[1]
    
\end{document}