% book example for classicthesis.sty
\documentclass[
  % Replace twoside with oneside if you are printing your thesis on a single side
  % of the paper, or for viewing on screen.
  %oneside,
  twoside,
  11pt, a4paper,
  footinclude=true,
  headinclude=true,
  cleardoublepage=empty
]{scrbook}

\usepackage{lipsum}
\usepackage[linedheaders,parts,pdfspacing]{classicthesis}
\usepackage{amsmath}
\usepackage{amsthm}
\usepackage{acronym}

\title{Tomography of Antarctica}
\author{Florian W\"olfl}

\begin{document}

\maketitle

%*******************************************************
% Abstract
%*******************************************************
% \pdfbookmark[1]{Abstract}{Abstract}
% \chapter*{Abstract}

In this Master thesis the first steps towards a continental scale seismic full waveform inversions 
using the adjoint method are performed. Additionally the Large Scale Seismic Inversion Framework 
(LASIF) is tested and optimized for inversions using the SPECFEM3D Globe solver.
The geographical region of interest for the inversion is Antarctica for several reasons:
(i) All other continents are already better studied with seismic tomographies.
(ii) In recent years more stations were installed on the continent so that the availability increased.
(iii) The coverage of Antarctica is quite well as the continent is surrounded by oceanic spreading zones 
and stations are installed on the surrounding continents.
(iv) The geology of Antarctica is not well known and a seimic tomography could give more information
for example on the orogenesis of the Transantarctic Mountains.

Waveforms are downloaded for 50 earthquakes and processed with LASIF tools. 
Synthetic waveforms are generated with SPECFEM3D and their misfit to the observed data is taken as
adjoint sources for the adjoint kernel simulations. 
The event kernels are then summed up to acquire the misfit kernel.
In further research the here completed steps could be used to update the initial velocity model and 
proceed with more iterations.

% This thesis also serves as guide on how to do adjoint simulations with SPECFEM3D Globe with the 
% help of LASIF.
 





\include{FrontBackMatter/dedication}
\include{FrontBackMatter/acknowledgements}
%*******************************************************
% Declaration
%*******************************************************
\pdfbookmark[0]{Declaration}{declaration}
\chapter*{Declaration}
\thispagestyle{empty}

Selbst{\"a}ndigkeitserkl{\"a}rung \\
Hiermit versichere ich, diese Masterarbeit selbst{\"a}ndig und lediglich unter Benutzung der angegebenen Quellen und Hilfsmittel verfasst zu haben. 
Ich erkl{\"a}re weiterhin, dass die vorliegende Arbeit noch nicht im Rahmen eines anderen Pr{\"u}fungsverfahrens eingereicht wurde.\\

\noindent Statement of authorship \\
With this statement I certify that this master thesis has been composed by myself. Unless otherwise acknowledged in the text, it describes my own work. All references have been quoted and all sources of information have been specifically acknowledged. This thesis has not been accepted in any previous application for a degree.  

Munich, August 3rd, 2015
\include{FrontBackMatter/contents}

\part{Introduction}

\begin{itemize}
\item What is inversion and tomography 
\item What is the adjoint method
\item Why Antarctica
\end{itemize}


\chapter{A Chapter}
\lipsum[1] % This is just some filler text. Remove before you start writing

\section{A Section}


\part{Data acquisition and processing}

For the selection of the earthquakes the built-in LASIF feature 
add\_gcmt\_events is used to select 50 events equally distributed in the
domain. The earthquakes cover a range of Moment Magnitudes between 5.2 and 6.3. 
No larger events were chosen as the point source approximation would not hold otherwise. 
The events are relatively equally distributed between 2005 and 2014. No events before 2005 can 
be downloaded by LASIF, which is not a problem as the instrumentation at before then was very scarce on Antarctica. 


\chapter{Yet Another Chapter}

  
\
\part{Simulations}
\chapter{ Chapter}
\lipsum[1]
    
\end{document}