% Set up the document
\documentclass[a4paper, 11pt, oneside]{Thesis}  % Use the "Thesis" style, based on the ECS Thesis style by Steve Gunn
%\graphicspath{{Figures/}}  % Location of the graphics files (set up for graphics to be in PDF format)

% Include any extra LaTeX packages required
\usepackage[ sort&compress]{natbib}  % Use the "Natbib" style for the references in the Bibliography % other option square, numbers, comma, 
\usepackage{verbatim}  % Needed for the "comment" environment to make LaTeX comments
\usepackage{vector}  % Allows "\bvec{}" and "\buvec{}" for "blackboard" style bold vectors in maths
\hypersetup{urlcolor=blue, colorlinks=true}  % Colours hyperlinks in blue, but this can be distracting if there are many links.
\usepackage{acronym}
\usepackage{tabularx}
% \usepackage{cite} % probably this causes more problems than it helps

%% ----------------------------------------------------------------
\begin{document}
\frontmatter	  % Begin Roman style (i, ii, iii, iv...) page numbering

% Set up the Title Page
\title  {Full Waveform Inversion of Antarctica}
\authors  {\texorpdfstring
            {\href{}{Florian W{\"o}lfl}}
            {Florian W{\"o}lfl}
            }
\addresses  {\groupname\\\deptname\\\univname}  % Do not change this here, instead these must be set in the "Thesis.cls" file, please look through it instead
\date       {\today}
\subject    {}
\keywords   {}

\maketitle
%% ----------------------------------------------------------------

\setstretch{1.3}  % It is better to have smaller font and larger line spacing than the other way round

% Define the page headers using the FancyHdr package and set up for one-sided printing
\fancyhead{}  % Clears all page headers and footers
\rhead{\thepage}  % Sets the right side header to show the page number
\lhead{}  % Clears the left side page header

\pagestyle{fancy}  % Finally, use the "fancy" page style to implement the FancyHdr headers

%% ----------------------------------------------------------------
% Declaration Page required for the Thesis, your institution may give you a different text to place here

\addtocontents{toc}{\vspace{1em}}  % Add a gap in the Contents, for aesthetics


% The Abstract Page
\addtotoc{Abstract}  % Add the "Abstract" page entry to the Contents
\abstract{
\addtocontents{toc}{\vspace{1em}}  % Add a gap in the Contents, for aesthetics

%*******************************************************
% Abstract
%*******************************************************
\pdfbookmark[1]{Abstract}{Abstract}
\chapter*{Abstract}

Short summary of the contents of your thesis.


}

\clearpage  % Abstract ended, start a new page
%% ----------------------------------------------------------------

\setstretch{1.3}  % Reset the line-spacing to 1.3 for body text (if it has changed)

%%--------------------------------------------------------------

\pagestyle{fancy}  %The page style headers have been "empty" all this time, now use the "fancy" headers as defined before to bring them back

%*******************************************************
% Table of Contents
%*******************************************************

%% ----------------------------------------------------------------
\lhead{\emph{Contents}}  % Set the left side page header to "Contents"
\tableofcontents  % Write out the Table of Contents

%% ----------------------------------------------------------------
\lhead{\emph{List of Figures}}  % Set the left side page header to "List of Figures"
\listoffigures  % Write out the List of Figures

%% ----------------------------------------------------------------
\lhead{\emph{List of Tables}}  % Set the left side page header to "List of Tables"
\listoftables  % Write out the List of Tables


% Abbreviations / Acronyms 
%% ----------------------------------------------------------------
\setstretch{1.5}  % Set the line spacing to 1.5, this makes the following tables easier to read
\clearpage  % Start a new page
%\lhead{\emph{Abbreviations}}  % Set the left side page header to "Abbreviations"
%\listofsymbols{ll}  % Include a list of Abbreviations (a table of two columns)
%{

%%%%%%%%%%%%%%%%
% There should not be a chapter but this should be done with the \lhead above, doesn't work though 
\chapter*{Acronyms}
\begin{acronym}[UML]
    \acro{CEM}{Comprehensive Earth Model}
    \acro{EA}{East Antarctic craton}
    \acro{FD}{Finite Differences}
    \acro{FDSN}{International Federation of Digital Seismograph Networks}
    \acro{GLL}{Gauss-Lobatto-Legendre}
    \acro{GSN}{Global Seismographic Network}
    \acro{IRIS}{Incorporated Research Institutions for Seismology}
    \acro{LASIF}{LArge Seismic Inversion Framework}
    \acro{MPI}{Message Passing Interface}
    \acro{SEM}{Spectral Element Method}
    \acro{TAM}{Transantarctic Mountains}
    \acro{WARS}{West Antarctic Rift System}
\end{acronym} 

%}




%% ----------------------------------------------------------------
%\clearpage  % Start a new page
%\lhead{\emph{Physical Constants}}  % Set the left side page header to "Physical Constants"
%\listofconstants{lrcl}  % Include a list of Physical Constants (a four column table)
%{
%% Constant Name & Symbol & = & Constant Value (with units) \\
%Speed of Light & $c$ & $=$ & $2.997\ 924\ 58\times10^{8}\ \mbox{ms}^{-\mbox{s}}$ (exact)\\
%
%}

%% ----------------------------------------------------------------
%\clearpage  %Start a new page
%\lhead{\emph{Symbols}}  % Set the left side page header to "Symbols"
%\listofnomenclature{lll}  % Include a list of Symbols (a three column table)
%{
%% symbol & name & unit \\
%$a$ & distance & m \\
%$P$ & power & W (Js$^{-1}$) \\
%& & \\ % Gap to separate the Roman symbols from the Greek
%$\omega$ & angular frequency & rads$^{-1}$ \\
%}
%% ----------------------------------------------------------------
% End of the pre-able, contents and lists of things
% Begin the Dedication page

%\setstretch{1.3}  % Return the line spacing back to 1.3
%
%\pagestyle{empty}  % Page style needs to be empty for this page
%\dedicatory{For/Dedicated to/To my\ldots}
%
%\addtocontents{toc}{\vspace{2em}}  % Add a gap in the Contents, for aesthetics


%% ----------------------------------------------------------------
\mainmatter	  % Begin normal, numeric (1,2,3...) page numbering
\pagestyle{fancy}  % Return the page headers back to the "fancy" style

% %*******************************************************
% Dedication
%*******************************************************
\thispagestyle{empty}
\pdfbookmark[1]{Dedication}{Dedication}

\vspace*{3cm}

\begin{center}
    To someone special 
\end{center}

% S%*******************************************************
% Acknowledgments
%*******************************************************
\pdfbookmark[1]{Acknowledgements}{acknowledgements}
\chapter*{Acknowledgements}




% %*******************************************************
% Table of Contents
%*******************************************************
\pdfbookmark[1]{\contentsname}{tableofcontents}

\setcounter{tocdepth}{2} % <-- 2 includes up to subsections in the ToC
\setcounter{secnumdepth}{3} % <-- 3 numbers up to subsubsections

\tableofcontents 

%*******************************************************
% List of Figures and of the Tables
%*******************************************************

%*******************************************************
% List of Figures
%*******************************************************    
\pdfbookmark[1]{\listfigurename}{lof}
\listoffigures

%*******************************************************
% List of Tables
%*******************************************************
\pdfbookmark[1]{\listtablename}{lot}
\listoftables
  
%*******************************************************
% List of Listings
%******************************************************* 
\pdfbookmark[1]{\lstlistlistingname}{lol}
\lstlistoflistings 
   
%*******************************************************
% Acronyms
%*******************************************************
\pdfbookmark[1]{Acronyms}{acronyms}
\chapter*{Acronyms}
\begin{acronym}[UML]
    \acro{CEM}{Comprehensive Earth Model}
    \acro{LASIF}{LArge Seismic Inversion Framework}
\end{acronym} 


%\part{Introduction}

%*******************************************************
% Introduction
%*******************************************************
% \pdfbookmark[0]{Introduction}{introduction}

\lhead{\emph{Introduction}}

\chapter{Introduction}

\section{Objective of this Thesis} % thesis in capital letters?

The objective of this work is to perform the initial steps needed for the improvement of a continental scale tomographic model for the crustal and upper mantle structure.
% For this a starting model like S20RTS within the Comprehensive Earth Model is used.
% by means of full seismic waveform inversion using the adjoint technique. 
Long period seismic signals of 60 to 120 seconds % check the numbers again
are used for the first iteration of the forward model. 
The limitation to only low frequencies and %number of iterations
iterations is due to the restricted time for the Master's thesis.
% The model updates are done by simulating a forward model and with its result an adjoint model. % improve this sentence
Testing the Large Scale Seismic Inversion Framework (LASIF) in general and with respect to its compatibility with SPECFEM3D 
GLOBE is a further intention of this work.
% The preprocessing and preparation for the adjoint simulations is performed with LASIF. 
The measured waveforms are dominated by surface waves. 
% are used for the tomography as 
They are more suitable for large upper mantle structures than body waves, as they are normally sensitive to a depth of a few 
hundred kilometers \citep{Morelli2004}.

\section{Overview of the Thesis}
In the following thesis the first steps of a seismic tomography with the full waveform inversion
(FWI) 
are described.
Reasons why Antarctica is chosen as region of interest are specified and the unique geologic 
situation of the southernmost continent is explained.
The framework LASIF and its tools to download and process the required 
waveforms are introduced.
Furthermore the Spectral Element Method (SEM) that builds the basis of the SPECFEM3D GLOBE 
solver is depicted.
The theoretical background of the FWI is shortly discussed and how model improvements 
of an initial Earth velocity model can be achieved with the adjoint method.
The Comprehensive Seismic Earth Model (CSEM) \citep{Afanasiev2015} used for the simulations is shown with
respect to differences to one-dimensional models for the Antarctic region.
The workflow of the simulations consisting of the mesh creation, forward simulations, 
misfit determination, computation of adjoint sources and the adjoint kernel simulations
are described.\\
Subsequently the resulting event and misfit kernels are presented and explained.
Encountered problems with the solver are discussed and possible next steps are proposed.
The thesis ends with a tutorial on how to perform adjoint kernel simulations
with LASIF and SPECEM3D GLOBE for future users.


\section{Short Overview of Seismic Tomography}

Tomography in seismology started with simple ray based methods.
The traveltime of one phase (often the first arrival) from an observed earthquake
is compared with the simulation with a one-dimensional model.
The first well-known one of the 1D models was the Preliminary Reference Earth Model
(PREM) by \citealp{dziewonski1981preliminary}.\\
%
Later with higher computing power multiple phases and surface waves were included
to improve the quality of the inversions.
Different 3D tomographies agree relatively well for long wavelengths, but much less
for higher frequencies \citep{Liu2012}. \\
%
Full waveform inversion are used for a long time in exploration geophysics;
this method does not require the picking of individual seismic phases.
As a larger part of the seismogram is used it is especially well suited
for regions with sparse data.
This is one of the reasons why FWI is chosen for this work.

\section{Suitability of Antarctica}

There are several reasons why the Antarctic continent is a suitable and interesting region 
to study with seismic tomography:
\begin{enumerate}[i]
\item All other continents are already better studied with seismic tomographies.
\item In recent years more stations were installed on the continent so that the availability 
of seismic waveforms increased.
\item The coverage of Antarctica is quite well as the continent is surrounded by oceanic spreading zones 
and stations are installed on the surrounding continents.
\item The geology of Antarctica is not well known and a seimic tomography could give more information
for example on the orogenesis of the Transantarctic Mountains.
\end{enumerate}


\section{Geology of Antarctica}
Antarctica is an appropriate choice for a seismic tomography study as it is the least seismically explored continent 
due to its remoteness and being surrounded by large oceans. 
Its lithosphere features some one-of-a-kind geological properties. 

Tectonically Antarctica is divided in two parts by the Transantarctic Mountains (TAM) spanning from the Ross Sea to the Weddel Sea.
The two sections are the West Antarctic Rift System (WARS) and the much larger and geologically older East Antarctic (EA) craton that 
was part of the Gondwana supercontinent \citep{Gupta2009}. 
% more on WARS (one of the largest continental rift zones) , or TAM not caused by contraction or subduction, ... from Morelli and references therein
The highest mountains in the TAM are up to 4500$\,$m high and the mountain chain is circa 3500$\,$km long \citep{Morelli2004}.
Locations of the two parts and the TAM can be seen in \autoref{ant-map}.
% origin of TAM 
The orogenesis % is this term here correct?
of the TAM is disputed, possible reasons are combinations of thermal uplift due to its proximity of the WARS continental rift, 
isostatic rebound after unloading of ice masses, flexural uplift after the breakup of two tectonic plates or 
the remains of a larger high plateau \citep{VanWijk2008}.
Model results from \citealp{VanWijk2008} suggest that the most likely reason for the uplift of the TAM is an extension along the 
boundary between the EA and Western Australia. This could also explain the origination of the nearby Wilkes depression.
Furthermore hotspots have be suggested to exist in Antarctica like under the active Erebus volcano \citep{Gupta2009}.

The size of the continent (larger than Europe) makes long period tomography suitable.  
Seismological tomography of Antarctica started with the construction of global models of wave velocity like \citealp{Woodhouse1984}.
It is known from previous tomographic analysis that the older, eastern part of the continent consists of seismically faster material than 
the younger Rift System in the West \citep{Morelli2004}.

Seismic research on Antarctica was for a long time limited by the few available seismometers at research stations on the continent. 
The worldwide installation of the GEOSCOPE network let to first maps of Antarctica's lateral heterogeneities \citep{Roult1994}.
Before 2007 seismic studies relied on global networks or temporarily installed seismometer networks, the most notable
example is the TAMSEIS experiment \citep{Lawrence2006}. 
After 2007 permanent broadband stations were installed within the Gamburtsev Antarctic Mountains Seismic Experiment (GAM-SEIS/AGAP)
as well as the Polar Earth Observing Network (POLENET ANET); 
in total there are now 77 year-round seismic stations available \citep{Anthony2014}.  




% maybe more on concepts of inversion and tomography
 
% Why FWI was chosen
% simple classical ray method -> finite frequency methods -> FWI (fully numerical)


\begin{figure}[h]
\begin{center}
\includegraphics[width = 1\textwidth]{images/antarctica-map_edit.png}
\caption[Map of Antarctica \url{http://lima.nasa.gov/pdf/A3_overview.pdf}.]
{This map shows the West Antarctic Rift System the Transantarctic Mountain Chain as well as the East Antarctic Craton 
(Landsat Image Mosaic of Antarctica \url{http://lima.nasa.gov/pdf/A3_overview.pdf} (modified)).}
\label{ant-map}
\end{center}
\end{figure}



% \section{Introduction to Inversion Methods}

% see Liu Gu 

%\section{A Section}


%\part{Data Acquisition and Processing}

%*******************************************************
% Data acquisition and processing
%*******************************************************
% \pdfbookmark[0]{Data Acquisition and Processing}{data_processing}
%\lhead{\emph{Data Acquisition}}

\chapter{Data Acquisition and Processing}

\section{Data Acquisition}

The domain is chosen so that enough stations are available from the continents that surround Antarctica.
It is centred at the South Pole and extends 115$^\circ$ in North-South direction and 110$^\circ$ in
East-West direction. 
For the selection of the earthquakes the built-in LASIF feature \
add\_gcmt\_events is used to select 50 events equally distributed in the
domain. %(see \autoref{eqmap}). 
Nearly six gigabyte of waveform data are acquired this way.
The earthquakes cover a range of Moment Magnitudes between 5.2 and 6.3 (\autoref{depth_scatter}). 
With 48 out of 50 most earthquakes having a depth smaller than 50\,km. Among these most 
events are much shallower than 50\,km. Only two events are in a depth range of 150 to 200\,km. 
No larger events are chosen as the point source approximation would not hold otherwise. 
The events are relatively equally distributed between 2005 and 2014 (\autoref{temp_dist}). 
No events before 2005 are used as the instrumentation before then was very scarce on Antarctica. 
\autoref{event_data} shows that on average more waveforms are available for the more recent earthquakes.
The coverage of the Antarctic continent by the selected data is visualised with a raydensity plot (\autoref{raydens}) 
that connects all stations with the recorded earthquakes. It can clearly be seen that the highest aggregation of stations 
exists in New Zealand. The reason therefore is the high seismic risk due to the proximity of the New Zealand population to tectonic plate boundaries (Alpine Fault and Kermadec Trench). \\


%\begin{figure}[H]
%\begin{center}
%\includegraphics[width = 0.85\textwidth]{images/eqmap.png}
%\caption{The red lines indicate the borders of the domain. The size of the ellipsoids depend on the magnitude of each event.}
%\label{eqmap}
%\end{center}
%\end{figure}

%\begin{figure}[H]
%\begin{center}
%\includegraphics[width = 1\textwidth]{images/magnitude_hist.eps}
%\caption{Moment magnitudes range from 5.2 to 6.3, while most events are smaller than 5.6.}
%\label{mag_dist}
%\end{center}
%\end{figure}

\begin{figure}[h]
\begin{center}
\includegraphics[width = 1\textwidth]{images/temporal_distribution_2.eps}
\caption[Temporal distribution of the earthquakes.]
{The histogram shows the number of earthquakes that were used from every year between 2005 and 2014.
The distribution of events is spread relatively well over the years.}
%{The earthquakes are equally spread over 
% interval from 2005 to 2014.}
\label{temp_dist}
\end{center}
\end{figure}

\begin{figure}[h]
\begin{center}
\includegraphics[width = 1\textwidth]{images/magnitude_vs_depth.png}
\caption[Depth distribution of the earthquakes.]{The earthquakes cluster between 10 and 20\,km, 
2 are between 150 and 200\,km. 
Histograms show the distribution of Moment Magnitude (right side) and depths (top).}
\label{depth_scatter}
\end{center}
\end{figure}

% PRELIMINARY PLOT FOR EVENT DATA
\begin{figure}[h]
\begin{center}
\includegraphics[width = 1\textwidth]{images/event_data_barchart_sorted_by_origin_time.png}
\caption[Raw seismograms per event]{Events sorted by origin time (from top to bottom) with the amount 
of acquired raw seismograms per event.}
\label{event_data}
\end{center}
\end{figure}

%\begin{figure}[H]
%\begin{center}
%\includegraphics[height = 0.95\textheight]{images/depth_histogram.eps}
%\caption{The earthquakes cluster between 10 and 20\,km, 2 are between 150 and 200\,km (not shown on this plot).}
%\label{depth_dist}
%\end{center}
%\end{figure}

\begin{figure}[H]
\begin{center}
\includegraphics[width = 1\textwidth]{images/raydensity_no_title.png}
\caption[The raydensity plot for the 50 events.]{The raydensity plot connects all stations (triangles) with the 50 
events (beachballs) for which waveforms exist. 
The 593 stations do not have data for every earthquake, as some events are far away so that the events cannot be well
distinguished from the instrument noise. In total there are more than 9500 ray paths.}
\label{raydens}
\end{center}
\end{figure}

As the station from the Global Seismographic Network (GSN) are the most important ones due to their high standards on used 
instruments and continuous recording, it is made sure that all available waveforms can be used in the LASIF project. 
Therefore data missing from the LASIF download is acquired from IRIS via an obspy fdsn client \citep{Krischer2015}. 
The waveform data is then horizontally correct rotated and StationXML files are downloaded from the IRIS website to
ensure the availability of instrumentation metadata for all required time intervals. \\

%After the data is collected it is checked if station information is available for each waveform, if duplicates exist, if 
%waveforms are corrupted and all raypaths from station to event location are within the domain. 
%If the last check fails an automatic script from LASIF is created to delete the violating raypaths. 
%When all the checks pass the waveforms can be preprocessed. 

After the collection of data the following checks are performed:
\begin{itemize}
\item Is station information for every waveform available?
\item Do duplicate waveforms exist?
\item Are waveform data files corrupted?
\item Are all waveforms within the LASIF domain?
\end{itemize}

Waveforms that do not pass the checks can be directly deleted with automatically created script.
After the checks all seismograms can be processed. 


%\lhead{\emph{Data Processing}}

\section{Data Processing}

The real waveform data and the simulated ones have to be processed as similarly as possible to avoid misfits only due
to different data manipulation. 
The preprocessing of the observed data starts with the creation of a new iteration for which the frequency band and 
the seismic solver is selected.
For all the iterations SPECFEM3D GLOBE CEM is selected as solver and a period band between 60 to 120 seconds is chosen.
Following iterations would then also include higher frequencies.

%\begin{table}[h]
%\begin{center}
%\begin{tabular}{{l|ccc}}
%Iteration number  &1    			&2 				&3 \\
%\hline
%Frequency band    &60-120\,seconds	&40-70\,seconds	&60-120\,seconds          \\
%\end{tabular}
%\end{center}
%\end{table}

Hereby the larger value for the period indicates the corner frequency for the highpass filter and the lower value the 
corner frequency for the lowpass filter. Hence the preprocessing can be understood as the application of a bandpass filter. 
Filtering is required not just to remove noise, but also to make the data comparable to the later created synthetic waveforms.
To make the following instrument response faster the sampling rate is downsampled from 20 to 10$\,$Hz.
Any linear trends in the data are removed and the mean of the data is set to zero.
%
As every seismic instrument has its own characteristics these have to be removed to make the different observed seismograms 
comparable within each other and to the synthetic seismograms.
%
Furthermore it is necessary that the original and synthetic datasets have the same length, therefore the seismogram is trimmed in case 
the observation period is too long or interpolated too make the number of data points in both files equal. 
For the adjoint method in SPECFEM displacement seismograms are needed and therefore all data has to be converted to 
displacement data as some seismograms were given as velocity data.

After the simulations are finished also the synthetic seismograms have to be adapted to a common sampling rate and 
they have to cover the same time interval as the observed data.
Furthermore the standard procedures demean and detrend are applied.
% Convolution with source time function











%\part{Simulations}

%*******************************************************
% Simulations
%*******************************************************
% \pdfbookmark[0]{Simulations}{simulations}
\lhead{\emph{Forward simulations}}

\chapter{The Spectral Element Method and Adjoint Simulations}

% The text about the SEM could also be part of the beginning 

\section{The Spectral Element Method}

\subsection{Why Spectral Elements}

Many different concepts have been used in the last decades for seismological simulations. 
While the Finite Difference (FD) Method is the easiest in terms of implementation, it is not suitable for 
complex geometries due to its regular grid. % (This should probably be explained better, problems with cubic elements and curved surfaces ...
Furthermore numerical dispersion errors limit the accuracy of this method.
The Pseudospectral Method offers much better accuracy, however the required global communications do not allow the 
implementation on parallel hardware. Without parallelisation large simulations like in this work are downright impossible.
This leaves then to choose between the Discontinuous Galerkin Method and the Spectral Element Method. As the latter one 
is already well established in the seismological community and the adjoint inversions have been performed before with the
SPECFEM software, I choose the SEM for this work. 
% DG harder to implement, SEM methods bit faster to run, should SeisSol be mentioned here? 
% This sentence here or in the next section? Also maybe with explanations for each advantage
The Spectral Element Method is characterised by its geometrical flexibility, small numerical errors (exact in the space domain), 
efficiency on parallel hardware and implicit free surface boundary conditions. 
The latter point makes this method especially suitable for surface wave simulation. % is this relevant? 

\subsection{Basic Concepts of the Spectral Element Method}

% How should Heiner's book be cited?

The equation that has to be solved with the SEM is the elastic wave equation. The one dimensional case is: 

\begin{equation}\label{1Dwave}
\rho(x) \ddot{u}(x,t) -  \partial_x \left[ \mu(x) \partial_x u(x,t) \right] = f(x,t)
\end{equation}

Where $\rho$ is the density, $u$ the unknown function (displacement), $\mu$ the shear modulus and $f$ the source or external forces.

The unknown function is approximated by superposing a finite amount of basis functions and the equation can then be written in matrix 
notation as follows: 

\begin{equation}\label{matrix_equation}
\underbrace{\textbf{M}}_{\text{Mass matrix}} \cdot \ \ddot{\textbf {u}} (t)  \ + \underbrace{\textbf{K}}_{\text{Stiffness matrix}} \cdot \ \textbf{u} (t) \, = \, \underbrace{\textbf{f} (t)}_{\text{Source term}} 
\end{equation}

%the mass matrix is defined as 
%\begin{equation} \label{mass}
%M_{ji} \ = \ \int_{G} \rho(x) \ \psi_j(x) \ \psi_i(x)  \ \mathrm{d}x
%\end{equation}
%
%the stiffness matrix is defined as 
%\begin{equation} \label{stiffness}
%K_{ji} \ = \ \int_{G} \mu(x) \ \partial_x \psi_j(x) \ \partial_x \psi_i(x) \ \mathrm{d}x
%\end{equation}

This equation is solved for every element in the domain. 
As the solution has to be found numerically the problem has to be discretised. 
The whole domain is therefore divided in mesh elements, which reflect the topography and internal discontinuities.
The elements can have different mechanical properties so that complex heterogeneous media can be modelled. 
Each element contains then an equal number of unevenly spaced (denser at the boundaries) grid points, the 
Gauss-Lobatto-Legendre (GLL) collocation points.
The interpolation with Lagrange polynomials as basis functions is exact at these grid points. 
%For each element in the domain the unknown function $u$ is exactly interpolated with Lagrange polynomials at unevenly spaced 
%(denser at the boundaries) grid points, the Gauss-Lobatto-Legendre (GLL) collocation points. 
The order of the Lagrange polynomials determines the number of collocation points per element and subsequently the accuracy of the simulation. 

Stress free boundary conditions are applied, which represent the stress free state at the Earth's surface. 

As also the numerical integration is performed by GLL quadrature with the use of Lagrange basis functions, the orthogonality of these polynomials 
leads to a diagonal mass matrix, which makes explicit extrapolation possible. This is the main reason for the efficiency of the Spectral Element Method.

The elemental solutions that overlap at their boundaries are then put together in a global system. 
This global set of equations is then extrapolated with a Finite Difference scheme.
Numerical errors of the FD method together with inaccuracies of the numerical integration determine the total error of the SEM.



\section{Forward and Adjoint simulations}


\subsection{Overview}
% reality vs. model
To improve the Earth model a measure is required how close the simulation is to the real world. 
% phase vs. amplitude
There are two possibilities to quantify the differences between the observed seismograms and the synthetic data.
One can either measure the traveltime phase differences or consider amplitude differences.
While the traveltime is quasi-linearly linked to structural heterogeneities (which we are interested in),
the measured amplitude is effected nonlinearly by them and also strongly dependent on the local geology 
around the seismometer \citep{Fichtner2008}.
For this reason I will focus on phase differences in this work.
% short intro to FWI
Owing to the application of the full waveform inversion (FWI) no specific seismic phases have to be identified 
individually \citep{Fichtner2011a}. 
% Furthermore the analyses for body and surface wave are naturally combined for the inversion. % Is that correct this way? See Fichtner2008 page 2
This would be required with classic tomography based on ray theory or finite frequency tomography that already 
accounts for 3D effects of wave propagation. % correct that way?
Whereas the two methods mentioned before can be solved (semi-)analytically, FWI is entirely based on numerical
solutions of the equations of motion.

% probabilistic vs. deterministic approach, latter one used in this work, see Fichtner book 

% minimisation of misfit function
The goal is now to minimise the misfit function, which can be achieved by setting its gradient to zero.
% adjoint method
The computation of the gradient can be achieved using the Adjoint Method. 
For this technique the residuals between the forward simulation and the observed data is used as source time function % REALLY
for the adjoint simulation. 
The adjoint simulation starts at the receiver and propagates backward in time towards the actual earthquake source.
From the interaction between the forward and the adjoint simulation the gradient can be obtained. 
% misfit kernel
Summing up the gradients for all events and receivers leads to a misfit kernel. 
This kernel provides information about the sensitivity to perturbations in the model \citep{Magnoni2012}.
With these results the Earth model can be updated and new simulations can be started to iteratively further improve the model.

In this thesis I try to identify the model parameters as good as possible (deterministic approach), instead of assigning
different probabilities to various possible parameter values as it is done with the probabilistic approach.

\subsection{The Adjoint Method}

% The adjoint method is described here in more detail

The forward simulations of the events are performed with SPECFEM3D GLOBE, which is described in more detail in the next section.
%
%adapt paragraph to section above according to Fichtner 2008
%MISFIT
With the results of the forward model the misfit can be computed.
The phase misfit used with the LASIF Framework follows \citealp{Fichtner2009}:
%
% see page 4 in Fichtner 2008
\begin{equation} 
E_p^n(u_i^0, u_i) \, := \, \int_{\mathbb{R}^2} W_p^2(t,\omega) [\phi_i(t,\omega) - \phi_i^0(t,\omega)]^2 
   \, \mathrm{d}t \, \mathrm{d}\omega
\end{equation}
%
With $W_p$ symbolizing a positive weighting function, $\phi_i^0$ the observed phase and $\phi_i$ the synthetic phase.
The difference $\Delta \phi_i = \phi_i - \phi_i^0$ corresponds to a time shift $\Delta t$. 
And the phase difference can then be expressed as $\Delta \phi_i = \omega \Delta t = 2 \pi f \Delta t$.
Due to ambiguity this quantity cannot be interpreted if it is larger than $\pi$ (a half period).
The misfit function is defined as an $L_2$ norm. % L2 norm as in Fichtner 2009, Fichtner 2008 does have the L_n norm

The synthetic and observed data is Fourier transformed into the time-frequency domain with the Gaussian window
$h(t) = (\pi \sigma)^{\frac{1}{4}} e^{\frac{-t^2}{2 \sigma^2}}$
and the window parameter $\sigma$, which is close to the dominant frequency of the waveform:

\begin{equation}
\tilde{u}^0 (t, \omega) = F_h[u_i^0](t,\omega) := \frac{1}{\sqrt{2\pi}} \int_{-\infty}^{\infty} u_i^0 h(\tau - t) e^{-i \omega \tau} \mathrm{d}\tau
\end{equation}


% Weighting function
As small disturbances in the wavefield can cause large variations in the phase especially for small amplitudes
a weighting function $W_p$ is required.
%
%Furthermore the purposes of a weighting function are to avoid phase discontinuities, select certain time intervals of
%a seismograms or weighting them higher or lower and to reduce noise.
A weighting function is constructed as the product of Gaussian windows in time ($W_T$) and frequency ($W_F$) 
and a logarithm of the displacement function to emphasize waveforms with small amplitudes.
With the temporal window the most relevant sections of a seismogram can be selected. 
The time windows are automatically selected with LASIF. 
The parameters for selection were tuned for this work.
Alternatively they can be manually selected with the LASIF misfit GUI \autoref{misfit_gui}.
The frequency window helps to reduce noise and the depth resolution can be improved by weighing large frequencies up.
Following \citealp{Fichtner2009} the phase weighting function has the following structure:

\begin{equation}
W_p(t, \omega) = W_T(t) W_F(\omega) \frac{  \log[1 + \vert \tilde{u}^0 (t, \omega) \vert ] }
{\max_{t,\omega} \log  [1 + \vert \tilde{u}^0 (t, \omega) \vert ]}
\end{equation}


%To avoid discontinuities at different frequencies with the phase difference a correlation function between
%the synthetic and observed frequencies and the Fourier transform of this function has to be computed; 
%this is described in more detail in \citealp{Fichtner2008}. % in more detail?
%However this procedure does not help if observations and simulation results are out phase an appropriate
%weighting function or suitable filter has to be applied. % Ist mir nicht so ganz klar

% the weighted phase change has to be bounded by $C||u||_2$ with $C<\infty$ \citealp{Fichtner2008}
% Approximation of $\Delta \phi$ by linear term of the Tayler series
% $W_p = |u~|$ so that phases with large amplitudes and frequencies close to the dominant period are emphasized
% $W_p$ as filter in the Time-Frequency domain


\begin{figure}[h]
\begin{center}
% Probably better image requird with East component
\includegraphics[width = 1\textwidth]{images/TF_domain_weight_func_book.png}
\caption{The differences between the observed and synthetic seismograms are multiplied with a weighting function,
as some waves are too fast and some are too slow.
This results in adjoint sources for the backward simulations.
%Central plots: Weighted phase differences.
%Right plots: The used weighting functions.
%time-frequency domain and the weighting function $W_p$. 
Graphic from \citealp{Fichtner2011}.}
\label{weight_func}
\end{center}
\end{figure}

\begin{figure}[h]
\begin{center}
% Probably better image requird with East component
\includegraphics[width = 1\textwidth]{images/misfit_gui.png}
\caption{Windows can be selected automatically or manually with the LASIF Misfit GUI.}
\label{misfit_gui}
\end{center}
\end{figure}



% siehe Magnoni seite 58

% noch veraltet

% Übergang zwischen den Sätzen
%The misfit function takes the least squares difference between the synthetics $\boldsymbol{s}$ and the 
%observed data $\boldsymbol{d}$ for the selected time window from $0$ to $\boldsymbol{T}$ for each of the three components. 
%(The variable naming follows the doctoral thesis from \citealp{Magnoni2012}).
%
%\begin{equation}
%F(m) \ = \ \frac{1}{2} \sum_{r=1}^N \int_0^T \lVert  s(x_r, t, m) - d(x_r, t)  \lVert^2  \mathrm{d}t
%\end{equation}
%
%The location of the $N$ receivers is expressed by $x_r$. $m$ is the currently used model and $t$ is the time instance in the window.



% Variation of misfit function delta F
% simplification with Born

%Variations in the misfit function are due to perturbations in the displacement field. 
%These can be linearised with the Born approximation \citep{Liu2012}.
%This approximation %is a first order approximation and 
%considers only one-time scattered wave paths; multiple scattering is ignored.

% Instead of the Born approximation and the reciprocity of the Green's function the adjoint problem can also be solved with 
% Lagrange multiplier approach (\citealp{Liu2012} and references therein).

% reciprocity of Green's tensor

For the adjoint simulation the waveform adjoint source is defined as the sum of the residuals between observed and 
synthetic data for all receivers $N$ evaluated at the receiver position $x_r$ with a delta function 
% this sentence has to be changed to make sense

\begin{equation}
f_i^{\dagger}(x,t) \ = \ \sum_{r=1}^N [ s_i(x_r, T-t) - d_i(x_r, T-t) ] \delta (x-x_r)
\end{equation}

It is computed with LASIF and is used to start the adjoint simulations.

% Formula of the adjoint source here

% With the residuals between observed and synthetic data used as virtual sources at the receivers the adjoint simulation can be started.

The results of the adjoint simulations make it possible to rewrite the variation of the misfit function.
The interaction between forward and adjoint simulation can be expressed as waveform misfit kernels (Fr\'{e}chet derivatives)
for density and elastic tensor variations 


The waveform misfit kernels (Fr\'{e}chet derivatives) can be computed with the results from the forward and the adjoint simulation. 

This way of computing the gradient of the misfit function is more efficient then for example with a finite difference 
approach \citep{Fichtner2006a}.

% minimisation of gradients 

 
The gradients from the adjoint simulations are then summed up to acquire event kernels. %????????

Summing up the event kernels gives a misfit kernel. The misfit kernel provides information about the sensitivity to perturbations in the model \citep{Magnoni2012}.

The Earth model can then be updated by applying steepest descent or conjugate gradient optimisation algorithms. 

% For the following reasons the steepest descent method is applied in this work 
% 10. Juli; Lion: conjugate gradient kann erst ab der 2. Iteration angewendet werden
% daher verwenden wir steepest descent wie in F. Magnoni beschrieben

% see Fichtner book for conjugate gradient method (page 128)


\section{Simulating with SPECFEM and Application of the Adjoint Method}

\subsection{Introduction to SPECFEM}

SPECFEM was started by Dimitri Komatitsch and Jean-Pierre Vilotte in Paris at the 
Institut de Physique du Globe (IPGP) \citep{Vilotte1998} in the nineties and later expanded by Jeroen Tromp and many others. 
It is used for seismic wave propagation simulations on global and regional scale and it is one of the 
most used community softwares for this purpose. 
It is mostly written in Fortran2003 and different versions are available for 2D or 3D use cases as well
as for regional or global simulation. 
In this work the version SPECFEM 3D GLOBE 7.0 is used.
The speciality of this version is its cubed sphere approach \autoref{cube}.
This means a cube consisting of smaller cubes is distorted into the shape of a sphere.
The advantage of this approach is that wave propagation in regular hexahedrons is easier to compute than
for example in variable shaped tetrahedrons. % This sentence should be revised
The whole globe is divided into 6 chunks that each consist of smaller slices.

%partionining of mesh and distribution with MPI (Message Passing Interface) on several CPUs 


\begin{figure}[h]
\begin{center}
\includegraphics[width = 0.6\textwidth]{images/cubed_sphere_large.png}
\caption{A large cube is built of smaller ones and then distorted into a sphere.
Graphic from the SPECFEM manual \citep{specfem_manual}.}
\label{cube}
\end{center}
\end{figure}


\subsection{Mesh Creation}

SPECFEM3D Globe comes with its own mesher, where the extent of the regional chunk of interest can be 
specified in a parameter file together with the number of slices in longitudinal and latitudinal direction.
It has to be checked if the created mesh in SPECFEM conforms well enough with the LASIF domain, 
otherwise the mesh has to be rotated around its central axis.
For this work the mesh and the domain corresponded well enough so that no rotations were required.
The timestep for the simulations is set to 0.14$\,$seconds.  
One of the six chunks of the globe is used for this regional mesh.
This chunk is subdivided into 16 slices along each edge, so 256 slices in total.
The number of slices determines on how many processors the mesh is created in parallel; 
it is the same number later used for the simulations. 
Every slice again contains smaller elements; the total number of elements in the mesh 
is 150,856. % elements, to be checked again for the final simulations

% depth extent

\begin{figure}[h]
\begin{center}
\includegraphics[width = 1\textwidth]{images/mesh_horizontal.png}
\caption{The created mesh from two perspectives. The 256 slices are visible on the surface.
The colors indicate the different boundary layers of the Earth. The inner core is only partially
included. }
\label{mesh}
\end{center}
\end{figure}


\subsection{Initial Earth model}

As initial model I take the velocity data from the data-comprehensive seismic Earth model (CSEM) \citep{Afanasiev2014}.
The tetrahedral mesh accounts for topography, bathymetry and small heterogeneities due to its variable resolution.
It is constructed as long-wavelength global three dimensional model and also contains well resolved regional models
like sedimentary basins or slab structures, where they are available.
The CSEM \autoref{CEM} is planned to be updated with tomographic models from research groups worldwide.
Various seismic solvers can use CSEM as initial model and it is suitable for traveltime ray tomography as well as
full waveform inversion.

A subset of the CSEM model is created with the same dimension as my SPECFEM mesh. 
The normalized differences of the CSEM velocities to the one-dimensional ak135 model \citep{Kennett1995} are shown for three
depths in \autoref{CEM}.

\begin{figure}[h]
\begin{center}
\includegraphics[width = 1\textwidth]{images/CEM_visualisation/vsh_overview.png}
%\caption{The CSEM model as used for my work. Vertically polarized P-wave velocity (top left), 
%vertically polarized S-wave velocity (top right), horizontally polarized P-wave velocity (bottom left),
%horizontally polarized S-wave velocity (bottom middle), density (bottom right).}
\caption{The normalized differences of the CSEM model \citep{Afanasiev2014} compared to the 1D ak135 model. 
Here the horizontally polarized S-wave velocity $v_{sh}$ are displayed. Red color indicates that the CSEM
velocities are smaller than the ak135 ones and blue symbolizes faster velocities.}  
% Colorbar!!
\label{CEM}
\end{center}
\end{figure}


\subsection{Preparation and Workflow for the Simulations}
% All simulations are run on SuperMUC, one of the World's fastest supercomputers at the Leibniz-Rechenzentrum in Garching. 
% More on SuperMUC ? 


% Preparation for multiple simulations
% it was not described in the Manual
% directory structure with script 

Two simulation runs are required for the computation of the traveltime sensitivity kernel.
%
At first the forward simulations are performed for all pairs of earthquake and receivers, for which observed seismograms are available.
The simulations are run in parallel on SuperMUC for all event-receiver pairs. 
Scripts to prepare the required directory structure are attached to this thesis. % Where ??????  CD, link 
One simulation is required for every event and seismograms are recorded at all stations. 
The state of the displacement, velocity and acceleration of the forward wavefield are saved for the later adjoint simulation.

% Window selection
In the next step an appropriate window, where observed and synthetic data agree well enough to be physically comparable,
is cut from the observed displacement seismograms and with the according processed section from the synthetic seismograms 
the adjoint sources are computed with LASIF.
The window can be automatically selected with LASIF. 
% Therefore the parameters for the selection like the Signal to Noise Ratio are adapted.
Alternatively the window can be selected with the Misfit GUI. 
A good procedure is to auto-select the windows and then check them manually in the LASIF GUI and 
add more windows if applicable.
For the event-receiver pairs with enough data this is done for all three components (North, East, Up).
% Calculation of Adjoint sources
% What part of the adjoint source computation is done with LASIF or SPECFEM?
With the prepared sources and the last state from the forward simulations the adjoint simulation can be started.
For all event-receiver pairs the synthetic earthquake starts now at the position of the receiver and propagates
backward in time.  
The forward simulation is back reconstructed from the saved last wavefield state. 
Its interaction with the adjoint simulation, which is propagating backwards in time,
yield the sensitivity kernels for the specific event. 
For every event the kernels for density $\rho$ as well as the $\alpha$ kernel for the P-wave velocity and the
$\beta$ kernel for the S-wave velocities are obtained. 
An example is given in the Results section \autoref{w_macq_kernel}.
Therefore SPECFEM3D convolves the adjoint with the forward wavefield.
The agreement between the adjoint displacement seismograms and the time-reversed forward seismograms should be very well. % testing
Differences can stem from numerical inaccuracies of the software. % correct?

The next step is to sum all 50 event kernels to one misfit kernel with the SPECFEM utilities.

% boundary kernels? 
% anisotropy? 

% summing of gradients for all event-receiver pairs

% update of Earth model

% iterations, which were not possible due to time constraints


\chapter{Results}

\section{Event kernels}

Isotropic event kernels are generated during the adjoint simulations from SPECFEM for density $\rho$, P-wave velocity ($\alpha$) 
and S-wave velocity ($\beta$). As example the beta kernels for one event southwest of Africa is shown in \autoref{sw_africa_beta_kernel},
\autoref{sw_africa_donut} and \autoref{sw_africa_banana}.

Smoothing is applied for every kernel to remove small scale effects that do not give information about physical 
properties of the Earth.
The effects of smoothing with 250$\,$km in horizontal and 5$\,$km in vertical direction can be seen
in \autoref{smoothing}.

\begin{figure}[h]
\begin{center}
\includegraphics[width = 1\textwidth]{images/kernels/sw_africa_smoothing_50km_beta_kernel_arrow.png}
\caption[Effects of smoothing on an event kernel]
{The $\beta$ event kernel at 50$\,$km depth of an event 
Southwest of Africa with magnitude 5.4 (2014-3-8).
The left side shows the not smoothed kernel and the
right side shows the kernel smoothed with 250$\,$km in 
horizontal and 5$\,$km in vertical direction.
The beachballs are shown to indicate the event location.
The shown surface is at a depth of 50$\,$km.}  
% Colorbar!!
\label{smoothing}
\end{center}
\end{figure}

% Write about depth dependent weighting 

%\begin{figure}[h]
%\begin{center}
%\includegraphics[width = 1\textwidth]{images/kernels/west_macquarie_event_kernel_100km.png}
%\caption{The $\alpha$ event kernel at 100$\,$km depth of an event West of the MacQuarie island with a moment magnitude of 6.1 (2006-11-16).}  
%% Colorbar!!
%\label{w_macq_kernel}
%\end{center}
%\end{figure}


%\begin{figure}[h]
%\begin{center}
%\includegraphics[width = 0.85\textwidth]{images/kernels/beta_kernel_southwest_of_africa_50km_depth_smooth_edit.png}
%\caption[Smoothed $\beta$ event kernel at 50$\,$km depth for one event]{The smoothed $\beta$ event kernel at 50$\,$km depth for an event 
%Southwest of Africa with magnitude 5.4 (2014-3-8). The beachball of the event is added as reference.}  
%% Colorbar!!
%\label{sw_africa_beta_kernel}
%\end{center}
%\end{figure}


\begin{figure}[h]
\begin{center}
\includegraphics[width = 1\textwidth]{images/kernels/beta_kernel_southwest_of_africa_donut_slice-90to90_smooth_depth_weight.png}
\caption[Slice from longitude -90$^\circ$ to 90$^\circ$ through the smooth $\beta$ event kernel for one event]{A slice from 
longitude -90$^\circ$ to 90$^\circ$ through the smooth $\beta$ event kernel for an event Southwest of Africa with 
magnitude 5.4 (2014-3-8). Depth dependent weighting is applied to enhance the structure of the kernel in greater depth.}  
\label{sw_africa_donut}
\includegraphics[width = 1\textwidth]{images/kernels/beta_kernel_southwest_of_africa_banana_slice_0to180_smooth_depth_weight.png}
\caption[Slice from longitude 0$^\circ$ to 180$^\circ$ through the smooth $\beta$ event kernel for one event]{A slice from longitude 
0$^\circ$ to 180$^\circ$ through the smooth $\beta$ event kernel for an event Southwest of Africa with magnitude 5.4 (2014-3-8).
 Depth dependent weighting is applied to enhance the structure of the kernel in greater depth.}  
\label{sw_africa_banana}
\end{center}
\end{figure}


\section{Misfit kernel}

The misfit kernel is the sum of all 50 event kernels. \autoref{sw_africa_s_india_msifit_beta_kernel} shows an example of a 
smoothed misfit kernel for two events (Southwest Africa and South Indian Ocean) in 50$\,$km depth.
Isosurfaces of the smooth misfit kernels can be visualized in 3D with ParaView, an example can be seen in \autoref{3d_misfit_sw_af}.

%\begin{figure}[h]
%\begin{center}
%\includegraphics[width = 0.85\textwidth]{images/kernels/smooth_misfit_kernel_sw_africa_s_india_50km.png}
%\caption[Surface of a smooth misfit kernel for two events]{A smoothed $\beta$ misfit kernel summed up from two events Southwest of Africa and the South Indian Ocean.
%The depth of the shown surface is 50$\,$km.}  
%% Colorbar!!
%\label{sw_africa_s_india_msifit_beta_kernel}
%\end{center}
%\end{figure}

\begin{figure}[h]
\begin{center}
\includegraphics[width = 0.85\textwidth]{images/kernels/misfit_kernel_sw_afr_s_indian_ocean.eps}
\caption[Surface of a smooth misfit kernel for two events]
{The smoothed $\beta$ kernels at a depth of 50$\,$km are shown for two events southwest of Africa and 
in the South Indian Ocean are shown. The Moment tensors are added for both events.
The addition of both kernels yields the misfit kernel (50$\,$km depth).
For visibility the colorbar range of the southwest of Africa event is ten times larger than for the event 
in the South Indian Ocean.
Therefore the event southwest of Africa clearly dominates the combined misfit kernel.}  
% Colorbar!!
\label{sw_africa_s_india_msifit_beta_kernel}
\end{center}
\end{figure}

%\begin{figure}[h]
%\begin{center}
%\includegraphics[width = 0.85\textwidth]{images/kernels/smooth_misift_sw_africa_s_india_val1e-7.png}
%\caption[3D isosurface of a misfit kernel for two events.]{Three dimensional isosurface of the smoothed $\beta$ 
%misfit kernel summed up from two events Southwest of Africa and the South Indian Ocean. 
%%The value of the isosurface is $1e-7 s/km^3$.
%}  
%% Colorbar!!
%\label{3d_misfit_sw_af}
%\end{center}
%\end{figure}


\begin{figure}[h]
\begin{center}
\includegraphics[width = 1\textwidth]{images/kernels/misfit_50_events_50km_depth.png}
\caption[Smooth misfit kernel for 50 events.]{Smooth misfit kernel added up from
all 50 events used in this work. The shown surface is 50$\,$km deep.}  
% Colorbar!!
\label{misfit_50}
\end{center}
\end{figure}



\begin{figure}[h]
\begin{center}
\includegraphics[width = 0.85\textwidth]{images/kernels/misfit_50events_3D.png}
\caption[3D isosurface of a misfit kernel for 50 events.]{Three dimensional isosurface of the smoothed $\beta$ 
misfit kernel summed up from all 50 events. Depth dependent weigthting is applied to enhance deeper structures.
%The value of the isosurface is $1e-7 s/km^3$.
}  
% Colorbar!!
\label{3d_misfit_50}
\end{center}
\end{figure}



\chapter{Conclusion}


\section{Encountered Problems}

Even though SPECFEM3D is relatively well documented compared to other solvers I used in 
earlier works, there are still some parts of the software that lack sufficient documentation.
For example multiple simultaneous runs of several earthquake events are possible with the 
software, however they are not mentioned in the manual.
Therefore I had to figure out the required files and directory structure from the error messages of
several runs. 
Another challenge was to find the correct time-shift needed for the adjoint simulations, as this 
one is different from the forward simulations, where no time-shift is applied.

It was planned to run all simulations with the option to undo attenuation effects exactly. 
This was no problem for the forward simulations, however it caused the adjoint kernel
simulations to run out of memory.
This problem is mentioned in the SPECFEM Manual and one possible reason is not exclusive use
of compute nodes. 
In this work I use the UNDO\_ATTENUATION parameter for the forward simulations to compute the 
misfits and the resulting adjoint sources as exactly as possible.
The adjoint kernel are then run with only partial physical dispersion enabled.

\section{Review of LASIF with SPECFEM3D Globe} 

One of the objectives of this work is to test how well LASIF interacts with the SPECFEM software,
as it has mostly been tested with SES3D before.
LASIF proves to be a large time-saver for the preparation of the simulations. 
The bulk of my observed data was downloaded with the download tools included in LASIF, which 
has the big advantage that all raypaths are checked if they are within the domain and can 
directly be deleted otherwise.
Only a few of the GSN seismic data had to be acquired from the IRIS webservices.
That the domain of this work covers the South Pole was an extra challenge, as the LASIF domain
are defined at the equator and then rotated to the center of the actual domain.
Lion Krischer did change the domain definitions, so that domains covering the poles  are now 
possible as well. 
The SPECFEM mesh is defined a little bit different at the margins of the domain, so that it 
covered additional area. However these effects are small enough to be neglected.
The processing of observed and synthetic data within LASIF worked well after a few changes to
the processing scripts.
As a large part of my synthetic seismograms are quite similar to their observed counterparts,
the window selection could be done mostly automatically. 
To select additional wiggles, especially the P-wave arrivals the LASIF misfit GUI proved helpful.
After small changes required for SPECFEM's naming convention the resulting adjoint source files
can be used without a problem in the solver.
So I can conclude that LASIF can be used for SPECFEM adjoint simulations and helps to save a 
lot of time compared to manual organisation of the whole project.



%------------------------------------------------
% The Acknowledgements page, for thanking everyone
\acknowledgements{
\addtocontents{toc}{\vspace{1em}}  % Add a gap in the Contents, for aesthetics

Thanks to my supervisors Heiner Igel and Lion Krischer. Also thanks to Michael Afanasiev for providing the CSEM.
I also want to thank Bernhard Schuberth for giving tips on the use of SPECFEM.
Furthermore I am grateful to the Leibniz-Rechenzentrum, which made it possible to run the simulations on the SuperMUC cluster.

}
\clearpage  % End of the Acknowledgements
%% 

%% ----------------------------------------------------------------
\label{Bibliography}
\lhead{\emph{Bibliography}}  % Change the left side page header to "Bibliography"
\bibliographystyle{unsrtnat}  % Use the "unsrtnat" BibTeX style for formatting the Bibliography
% \bibliographystyle{apalike}
\bibliography{./bibtex/library}  % The references (bibliography) information are stored in the file named "Bibliography.bib"

%% ----------------------------------------------------------------
% Now begin the Appendices, including them as separate files

\addtocontents{toc}{\vspace{2em}} % Add a gap in the Contents, for aesthetics

\appendix % Cue to tell LaTeX that the following 'chapters' are Appendices

% Appendix A

\chapter{Appendix}
\label{AppendixA}
\lhead{Appendix A. \emph{Appendix Title Here}}

% Write your Appendix content here.	% Appendix Title

\addtocontents{toc}{\vspace{2em}}  % Add a gap in the Contents, for aesthetics
\backmatter


%*******************************************************
% Declaration
%*******************************************************
\pdfbookmark[0]{Declaration}{declaration}
\chapter*{Declaration}
\thispagestyle{empty}

Selbst{\"a}ndigkeitserkl{\"a}rung \\
Hiermit versichere ich, diese Masterarbeit selbst{\"a}ndig und lediglich unter Benutzung der angegebenen Quellen und Hilfsmittel verfasst zu haben. 
Ich erkl{\"a}re weiterhin, dass die vorliegende Arbeit noch nicht im Rahmen eines anderen Pr{\"u}fungsverfahrens eingereicht wurde.\\

\noindent Statement of authorship \\
With this statement I certify that this master thesis has been composed by myself. Unless otherwise acknowledged in the text, it describes my own work. All references have been quoted and all sources of information have been specifically acknowledged. This thesis has not been accepted in any previous application for a degree.  


\end{document}  % The End
%% ----------------------------------------------------------------